\documentclass[a4paper,11pt,openright]{book}
\usepackage[swapnames]{frontespizio}
%\usepackage[classica]{topfront}
%\usepackage[pdftex]{graphicx} %per poter inserire le figure
%\usepackage[a4paper,left=2cm,bottom=2cm,right=2cm,top=2cm]{geometry}
\usepackage{amssymb,amsmath,amsthm,amsfonts}
\usepackage{xspace}
\usepackage{indentfirst}
\usepackage{graphicx}
\usepackage{tabularx}
\usepackage[pdfa]{hyperref}
\usepackage[noabbrev,capitalize, italian]{cleveref}
\usepackage{subfigure}
\usepackage[small]{caption}
%\usepackage{subcaption}
\usepackage{eucal}
\usepackage{eso-pic}
\usepackage{url}
\usepackage{booktabs}
\usepackage{afterpage}
\usepackage{parskip}
\usepackage{bookmark}
\usepackage{listings}
\usepackage{fancyhdr}
\usepackage{textcomp}
\usepackage{cite}
\usepackage{multirow}
\usepackage{adjustbox}
\usepackage[utf8]{inputenc}   %per riuscire a scrivere gli accenti
\usepackage[italian, english, polish]{babel}   %per riuscire a scrivere gli accenti
\usepackage{setspace}


%%%%%%%%%%%
\newenvironment{abstract}%
{\cleardoublepage%
\thispagestyle{empty}%
\null \vfill
\begin{center}%
\Huge \bfseries \abstractname 
\end{center}}%
{\vfill\null}
%%%%%%%%%%%%%%%%%

%%%%%%%%%%%%%%%%%%%%%%%%%%%%%%%%%%%%%%%%%%%%%

% CONFIGURAZIONE LINK E RIFERIMENTI
\hypersetup{%
    pdfpagemode={UseOutlines},
    bookmarksopen,
    pdfstartview={FitH},
    colorlinks,
    linkcolor={black}, %COLORE DEI RIFERIMENTI AL TESTO
    citecolor={blue}, %COLORE DEI RIFERIMENTI ALLE CITAZIONI
    urlcolor={blue} %COLORI DEGLI URL
}

%%%%%%%%%%%%%%%%%%%%%%%%%%%%%%%%%%%%%%%%%%%%%%

\frenchspacing

%%%%%%%%%%%%%%%%%%%%%%%%%%%%%%%%%%%%%%%%%%%%%%

% INFORMAZIONI PDF - PERSONALIZZARE
\pdfinfo{%
  /Title    (Caratterizzazione spettroscopica del sistema di Didymos dopo l'impatto con la sonda DART)
  /Author   (Gabriele Bertinelli)
  /Subject  (Spettroscopia asteroide)
  /Keywords (spettroscopia, asteroide, Didymos, dart, rotazione)
}

%%%%%%%%%%%%%%%%%%%%%%%%%%%%%%%%%%%%%%%%%%%%%%





\begin{document}
\selectlanguage{italian}
\begin{frontespizio}

    \Universita{Padova}
    
    
    \Dipartimento{Fisica e Astronomia "Galileo Galilei"}
    
    
    \Corso[Laurea]{Astronomia}
    \Logo[3.0cm]{figure/UNIPD.png}
    
    
    {\Titoletto{Tesi di laurea triennale}}
    \Titolo{Caratterizzazione spettroscopica del sistema di\\ Didymos dopo l'impatto con la sonda DART}
    
    
    \Relatore{Prof.ssa Monica Lazzarin}
    \NRelatore{Relatrice}{}
    
    \Correlatore{Prof.ssa Fiorangela La Forgia}
    \NCorrelatore{Correlatrice}{}
    
    
    \Candidato{Gabriele Bertinelli}
    
    
   
    \Annoaccademico{2021-2022}
    
\end{frontespizio}



\frontmatter

\vspace*{\stretch{1}}
\begin{flushright}
\noindent
\textit{Alla mia famiglia}
\end{flushright}
\vspace*{\stretch{6}}
\cleardoublepage

\vspace*{\stretch{1}}
\begin{flushright}
\noindent
\textit{"Complete the motion if you stumble"}\\
\textit{RHCP}
\end{flushright}
\vspace*{\stretch{6}}
\cleardoublepage


\begin{abstract}    
Qui ci sarà il riassunto della tesi.\\
16 novembre 2022 è la data dell’inizio di una nuova era di missione spaziali: l’era delle missioni Artemis, che si ripromettono di riportare sulla Luna l’essere umano.

Alle 07:47, ora italiana, il razzo Space Launch System di NASA ha illuminato a giorno il pad 39-B al Kennedy Space Center, dando inizio alla missione Artemis I.
Scopo della missione è testare SLS nel suo complesso, razzo, capsula, sottosistemi, procedure ecc. 
Dopo 26 giorni di missione, l’11 dicembre, la capsula Orion rientrerà nell'Oceano Pacifico, al largo delle coste di San Diego.

Questa mattina, dopo qualche problema nei centri di controllo, il lancio è stato nominale: i booster migliorati, ereditati dalle missioni Shuttle, hanno performato bene; il core centrale di SLS (il cilindro arancione) idem; i 4 pannelli solari del Service Module, costruito dall’ESA (con grande contributo italiano) si sono aperti nel modo corretto e stanno generando la corrente elettrica necessaria.
Attorno alle 09:14 è iniziata la Trans Lunar Injection Maneuver, manovra per spostarsi dall'orbita terrestre a quella lunare, durata 18 minuti (un record per il motore RL10). Tutto è andato come previsto e Orion è in rotta verso la Luna.

I prossimi step sono la separazione dell'ICPS, il 2o stadio che ha permesso le ultime due manovre, correzioni dell’orbita e il rilascio di 13 cubesat, tra cui ArgoMoon: cubesat costruito da Argotec e ASI, azienda aerospaziale italiana, con l’obiettivo di monitorare da vicino tutte le operazioni che compirà la Orion attorno alla Luna.

Nei prossimi giorni seguiranno aggiornamenti sulla missione.
\end{abstract}



\tableofcontents

\listoftables

\listoffigures


\mainmatter

\chapter{Corpi minori del Sistema Solare}\label{ch:ch_1}

%\section{Proprietà generali}

Dal 1801, quando Giuseppe Piazzi scoprì il primo corpo minore Cerere, categorizzato poi come asteroide, per sottolineare le differenze apparenti rispetto alle comete, si sono scoperti e categorizzati più di 1 milione di asteroidi\footnote{\href{https://solarsystem.nasa.gov/asteroids-comets-and-meteors/asteroids/overview/}{https://solarsystem.nasa.gov/asteroids-comets-and-meteors/asteroids/overview/}} La maggior parte orbita tra Marte e Giove e forma quella che è chiamata Main Belt (2-3.3 AU).

Esistono numerosi gruppi di asteroidi, a diverse distanze dal Sole. Nella \cref{tab:asteroid-classification} sono riportati i principali gruppi e famiglie, con le loro classificazioni dinamiche.

In questo capitolo verranno descritte brevemente le principali caratteristiche. La loro definizione è spesso legata alla risonanza con Giove o Nettuno, mentre altri gruppi sono definiti dall'intersezione della loro orbita con quella di un pianeta (i.e. Near Earth Objects e Mars Crossers).

\section{Distribuzione e classificazione dinamica}
In questo capitolo verrà decritta la classificazione in gruppi e famiglie.
La classificazione è riassunta nella \cref{tab:asteroid-classification} .\\
I corpi minori sono divisi in gruppi e famiglie in base alle loro caratteristiche orbitali. 
Groups are relatively loose dynamical associations, whereas families are tighter e sono probabilmente il risultato di una distruzione catastrofica di un asteroide “genitore” (in inglese è tutto così bello “parent asteroid”).

Per la classificazione è utile conoscere le definizioni analitiche di afelio e perielio di un'orbita:
\begin{equation}
    \begin{cases}
        Q=a(1+e) &\text{afelio}\\
        q=a(1-e) &\text{perielio}
    \end{cases}
\end{equation}

dove \textit{a} e \textit{e} sono rispettivamente il semiasse maggiore e l'eccentricità dell'orbita.

\paragraph*{Near Earth Objects}\label{sec:neo}
I Near Earth Objects (NEOs) rappresentano un gruppo eterogeneo di asteroidi (NEAs) e nuclei cometari estinti (NECs) che hanno orbite con un perielio minore di quello di Marte: $q<1.3$ AU. Si ritiene che i NEOs siano la principale fonte di meteoriti che arrivano sulla Terra.\\
In base alla relazione tra la loro orbita e quella della Terra sono catalogati in diversi sotto-gruppi: \textit{Atens} e \textit{Apollo} sono Earth Crossers (i.e. intersecano l’orbita della Terra), mentre gli \textit{Amors} si avvicinano alla Terra ma non intersecano mai la sua orbita. Gli Atens hanno un semiasse maggiore più piccolo di quello della Terra ($a<1$ AU) mentre gli Apollo hanno $a>1$ AU.


\begin{table}
    
    \begin{center}
    \resizebox{\linewidth}{!}{%
    
    \begin{tabular}{cccccc}
      \toprule
      \toprule
      \bf{Popolazione} & \bf{Famiglia/Gruppo} & \bf{a}     & \bf{q}     & \bf{Q}     & \bf{Commento} \\
      \midrule
      NEAs        &                 &       &       &       &           \\
                  & Atens           & $<1$  &              & $\geq 0.983$ & Earth Crossers     \\
                  & Apollos         & $>1$  & $\leq 1.017$ &              & Earth Crossers \\
                  & Amors           & $>1$  & $(1.017,1.3)$ &             &  \\
      \midrule
      Mars Crossers &               &       & $[1.3,1.67]$ &             &  \\
      \midrule
      \midrule
      \bf{Popolazione} & \bf{Famiglia/Gruppo} & \bf{a}     & \bf{e}     & \bf{i}     & \bf{Risonanze} \\
      \midrule
      Main Belt Asteroids &       &       &       &       &  \\
            & Hungarians & $[1.78,2.06]$ & $<0.18$ & $[16,34]$ & $[5:1, 4:1]$ \\
            & Inner MB & $[2.06,2.50]$ &       & $[18,32]$ & $[4:1,3:1]$ \\
            & Middle MB & $[2.50,2.82]$ &       &       & $[3:1,5:2]$ \\
            & Outer MB & $[2.82,3.28]$ &       &       & $[5:2,2:1]$ \\
            & Cybeles & $[3.28,3.70]$ & $<0.3$ & $<25$ & $[2:1,5:3]$ \\
            & Hildas & $\sim 3.97$ & $>0.07$ & $<20$ & $3:2$ \\
      \midrule
      Jupiter Trojans &       & $\sim 5.20$ &       &       & $1:1$ \\
      \midrule
      Centaurs &       & $[5.4,30]$ &       &       &  \\
      \midrule
      Neptune Trojans &       & $\sim 30$ &       &       & 1:1 Nep \\
      \midrule
      TNOs (o KBOs) &       & $>32$ &       &       & $[4:5, 2:3]$ \\
            &       &       &       &       & $[3:5, 1:2]$ \\
      \bottomrule
      \bottomrule
    \end{tabular}}%
    \end{center}
    \caption[Lista e definizioni delle principali popolazioni di asteroidi]{Lista e definizioni dei principali gruppi e famiglie di asteroidi. Il semiasse maggiore $a$, il perielio $q$ e l'afelio $Q$ sono espressi in AU; l'inclinazione $i$ in gradi. $e$ è l'eccentricità. Le risonanze, dove non è indicato, si riferiscono all'orbita di Giove.}
    \label{tab:asteroid-classification}
\end{table}%
  

\paragraph*{Mars Crossers}
Come fa intendere il nome, i \textit{Mars Crossers} sono oggetti che intersecano l’orbita di Marte.\\
È stato mostrato che la popolazione dei Mars Crossers, che è circa 4 volte più grande di quella dei NEOs, è rifornita da risonanze diffusive nella Main Belt (Migliorini et al. 1998; Morbidelli \& Nesvorny 1999; Michel et al. 2000; Bottke et al. 2002), dalla regione chiamata “intermediate-source Mars-crossing region” o IMC.

\paragraph*{Main Belt Asteroids}
I Main Belt Asteroids (MBA) sono asteroidi situati tra le orbite di Marte e Giove. La maggior concentrazione si trova tra 2.0 e 3.3 AU.\\
A causa delle risonanze orbitali dovute all’influenza gravitazionale di Giove, si vengono a creare molti gruppi, sotto-gruppi e famiglie. Di seguito sono riportati i più popolosi.

\qquad \textit{Hungaria}: prendono il nome da (434) Hungaria. Hanno un semiasse maggiore tra 1.78 e 2.06 AU, un’eccentricità minore di 0.18 e un’inclinazione tra 16° e 34°. 

\qquad \textit{Inner Main Belt}: hanno un semiasse maggiore tra 2.06 e 2.50 AU, limiti definiti dalle risonanze di moto medio 4:1 e 3:1.

\qquad \textit{Middle Main Belt}: hanno un semiasse maggiore tra 2.50 e 2.82 AU. I limiti sono definiti dalle risonanze di moto medio 3:1 e 5:2.

\qquad \textit{Outer Main Belt}: hanno un semiasse maggiore tra 2.82 e 3.28 AU. All’interno del gruppo troviamo le famiglie \textit{Koronis}, \textit{Eos} e \textit{Themis}.

\qquad \textit{Cybele}: hanno un semiasse maggiore tra 3.28 e 3.70 AU. Prendono il nome da (65) Cybele.

\qquad \textit{Hilda}: hanno un semiasse maggiore medio tra 3.70 e 4.20 AU. Prendono il nome da (153) Hilda.

\paragraph*{Jupiter Family Comets}
Le comete sono divise in due gruppi: comete a lungo periodo e comete a corto periodo.\\
Le comete a corto periodo hanno un periodo orbitale $< 20$ anni a basse inclinazioni. Hanno anche $2<T_J<3$, con $T_J$ Invariante di Tisserand, espressa da

\begin{equation}
    T_J=\frac{a_J}{a}+2\cos(i)\biggl[\frac{a}{a_J}(1-e)\biggr]^{1/2}
\end{equation}

dove $a$ e $a_J$ sono i semiassi maggiori dell'oggetto e di Giove rispettivamente ed $e$ è l'eccentricità dell'oggetto.\\
Visto che la loro orbita è controllata da Giove sono chiamate Jupiter Family Comets (JFCs). Si pensa che le comete a corto periodo provengano dalla Kuiper Belt, una grande riserva di piccoli corpi ghiacciati che si trova oltre Nettuno. A causa di collisioni e/o perturbazioni gravitazionali alcuni oggetti della Kuiper Belt riescono a fuggire dalla stessa e dirigersi verso il Sistema Solare interno. La successiva interazione con Giove può far sì che questi oggetti assumano parametri orbitali simili a quelli degli asteroidi.

\paragraph*{Centaurs}
I Centaurs sono oggetti che orbitano tra Giove e Nettuno ($5.4<a<30$ AU), ma non hanno ancora una definizione dinamica univoca.\\
Studi dinamici delle loro orbite indicano che i centaurs sono probabilmente degli oggetti con caratteristiche orbitali intermedie tra quelli della Kuiper Belt e le JFCs.\\
Sono probabilmente più simili a comete che ad asteroidi, e uno di questi, Chiron, è stato osservato avere attività cometaria.

\paragraph*{Trans Neptunian Objects}
Sono oggetti con $a\geq 30$ AU e sono divisi nei seguenti gruppi:

\qquad \textit{Kuiper Belt Objects}: si trovano tra 41 e 47 AU. A sua volta si divide nei sotto-gruppi dei \textit{Plutinos}, in risonanza 3:2 con Nettuno come Plutone e dei \textit{Cubewanos}, conosciuti come i classici KBOs.

\qquad \textit{Scattered-Disk Objects (SDOs)}: hanno orbite molto grandi e molto ellittiche, probabilmente dovute a un’interazione con Nettuno.

\qquad \textit{Oort Cloud Objects (OCOs)}: la Oort Cloud è una shell sferica di oggetti ghiacciati, di cui non si ha evidenza diretta perché troppo lontana e scura ma ipotizzata come luogo di origine delle comete a lungo/lunghissimo periodo. Si trova tra circa 50000 e 100000 AU.

\section{Potentially Hazardous Asteroids}\label{sec:pha}
Tra i NEO esiste una categoria di oggetti che riveste un ruolo importante: sono gli asteroidi potenzialmente pericolosi, Potentially Hazardous Asteroids (PHA). Rientrano in questa categoria tutti gli oggetti la cui minima distanza all'intersezione dell'orbita terrestre (Minimum Orbit Intersection Distance - MOID) è inferiore a 0.05 AU e la cui magnitudine assoluta H è minore di 22. Sono, quindi, quegli asteroidi che se impattassero con la Terra provocherebbero danni su larga scala, anche globale. Ad oggi si conoscono più di 2500 PHA, la maggior parte dei quali sono della famiglia Apollo, e ogni giorni vengo aggiunti o rimossi oggetti sulla base di nuovi risultati sulla base dei parametri orbitali.\\
La determinazione dell'orbita avviene sulla base delle osservazioni disponibili ed è definita da sei parametri orbitali, influenzati da diversi fattori. Quindi non si può determinare immediatamente con sufficiente accuratezza l'orbita ma servono osservazioni costanti nel tempo.\\
Al momento della scoperta di un PHA si determina l'orbita per i successivi 100 anni. Si possono determinare in tal modo i passaggi ravvicinati (fly-by) che l'asteroide avrà con la Terra e, conoscendo la regione di incertezza, la probabilità di collisione con la stessa.

\begin{figure}
    \centering
    \includegraphics[scale=0.35]{figure/pha_map.jpg}
    \caption[Visualizzazione delle orbite dei PHA conosciuti.]{Visualizzazione delle orbite dei PHA conosciuti. Si nota come la zona intorno all'orbita terrestre sia molto popolata. (NASA/JPL-Caltech)}
    \label{fig:pha_map}
\end{figure}

I PHA con una probabilità non nulla di impatto con la Terra nei futuri 100 anni vengono catalogati nella Sentry Risk Table\footnote{\href {https://cneos.jpl.nasa.gov/sentry/}{https://cneos.jpl.nasa.gov/sentry/} }, un sistema informatico gestito dal CNEOS (Center for Near-Earth Object Studies).\\
Oltre alla probabilità di collisione e alcune caratteristiche dell’oggetto (magnitudine assoluta, diametro) nella \cref{tab:asteroid-classification} vengono riportati i valori dell’asteroide sulla Scala Torino e Scala Palermo. Queste sono due classificazioni che permettono di quantificare il pericolo associato ad ogni asteroide: la Scala Torino è stata pensata per la comunicazione al pubblico, la Scala Palermo è più tecnica e viene usata direttamente dagli astronomi.
%%direi che ha poco senso mettere due paragrafi sulla scala milano e torino, ma non costa nulla

\chapter{Spettroscopia}\label{ch:ch_2}

\section{Introduzione}
Lo studio della mineralogia superficiale di singoli asteroidi o di gruppi di asteroidi può fornire i dati per migliorare la nostra comprensione della loro origine ed evoluzione. La superficie degli asteroidi può essere studiata dall'interpretazione delle proprietà osservabili per determinare la presenza, l'abbondanza e la composizione mineralogica.\\
La spettroscopia di riflettanza nel visibile (VIS) e nell'infrarosso (IR) viene ampiamente usata per determinare la composizione degli asteroidi poiché può caratterizzare la composizione superficiale della maggior parte dei tipi di asteroidi. Le caratteristiche diagnostiche negli spettri, che derivano da transizioni elettroniche e vibrazionali all'interno dei minerali e delle molecole, sono riscontrabili nell'intervallo di frequenza $0.35-2.50\,\mu m$.\\
I minerali più importanti presenti negli spettri degli asteroidi sono: olivine, pirosseni, metalli Ferro-Nichel, feldspati e fillosilicati idrati e composti organici.\\
La maggior parte degli asteroidi è composta da un mix di questi minerali. Poiché i parametri spettrali dei diversi assorbimenti (i.e. la posizione delle bande e il rapporto tra esse) sono legati a una specifica composizione del singolo minerale, l'analisi spettrale della superficie degli asteroidi è in grado, nella maggior parte dei casi, di rilevare le firme mineralogiche caratteristiche di una particolare specie.\\
La possibilità di rilevare una feature dipende dall'abbondanza della particolare specie, in modo che la forza della feature possa essere rilevata sopra il rumore dello spettro.\\
In spettri di alta qualità può essere anche determinata la composizione e le relative abbondanze di un mix di minerali.\\
Per questi motivi l'analisi degli spettri può fornire una varietà di dati sulla composizione della superficie degli asteroidi.

\section{Spettro degli asteroidi}
Il flusso incidente che arriva sulla superficie di un asteroide è diviso in due contributi: la parte riflessa e la parte assorbita, il cui rapporto dipende dall'albedo. La parte assorbita riscalda la superficie del corpo. Questo emette, di conseguenza, radiazione di corpo nero determinata dalla temperatura raggiunta. Quindi la Terra riceve sia il flusso "solare" riflesso dall'oggetto che la radiazione di corpo nero.\\
Negli spettri degli asteroidi la radiazione solare riflessa domina nel range che va dall'ultravioletto (UV) al vicino infrarosso (NIR): $0.35-2.50\,\mu m$; mentre a lunghezze maggiori ($2.5-5.0\,\mu m$) il contributo della radiazione di corpo nero dell'asteroide diventa rilevabile.

\section{Tipi tassonomici}
La tassonomia è la classificazione degli oggetti in categorie definite da alcuni parametri caratterizzanti. Sin dagli anni '70 diversi autori hanno creato diverse classi tassonomiche basandosi su proprietà osservabili.\\
La classificazione tassonomica più utilizzata è quella di Tholen (1984), basata su più di 400 spettri ricavati dalla Eight-Color Asteroid Survey (ECAS), nel range spettrale $0.3-1.1\,\mu m$. Dagli spettri ECAS si possono distinguere undici classi spettrali (A, B, C, D, F, G, Q, R, S, T e V).\\
Barucci et al., nel 1987, usarono sette colori spettrofotometrici e gli albedo ricavati dalle survey IRAS per definire nove classi spettrali e per ognuna delle sottoclassi identificate dallo studio dettagliato degli spettri.\\
Infine, Bus e Binzel (Bus \& Binzel (2002b), Bus \& Binzel (2002a)) hanno derivato la loro classificazione tassonomica da dati spettrofotometrici recenti.

\subsection{Tassonomia di Tholen}
Ognuna delle tassonomie menzionate sopra separa gli asteroidi in diverse classi secondo riflettività e albedo visuali (nel visibile) simili. Sebbene non viene usato nessun criterio mineralogico nel definire le classi, la tassonomia può essere usata per derivare la caratterizzazione mineralogica di ogni asteroide nelle varie classi.\\
Di seguito descriveremo brevemente le principali caratteristiche di ogni classe della tassonomia di Tholen e la loro relazione con le meteoriti.

\paragraph*{Classe A}
Gli asteroidi di tipo A mostrano una forte decrescita nella riflettanza attorno a $0.7\,\mu m$, relativa a composti metallici, e un forte banda di assorbimento centrata attorno a $1\,\mu m$ e nessun assorbimento rilevante a $2\,\mu m$ (caratteristica dei pirosseni).\\
Sono asteroidi molto rari, situati principale nella Main Belt interna. Si ritiene che questi oggetti siano frammenti del mantello differenziato di corpi progenitori di dimensioni maggiori.\\
Le analoghe meteoriti sono le acondrite a olivina e/o pallasite.

\begin{figure}
\centering
\includegraphics[scale=0.3]{figure/spettro_qsa.jpg}
\caption[Esempio di spettri di tipo Q, S e A.]{Esempio di spettri di tipo Q, S e A. Gli spettri sono normalizzati a $0.55\,\mu m$. Tra parentesi è riportato il range di albedo in percentuale. (Magrin, 2006)}
\label{spettro_qsa}
\end{figure}

\paragraph*{Classe S}
Rappresenta la seconda classe più numerosa ed è caratterizzata da spettri con una pendenza ripida attorno a $0.7\,\mu m$ e assorbimenti moderati attorno a $1$ e $2\,\mu m$. La loro albedo varia tra $0.1$ e $0.3$. Lo spettro tipico degli asteroidi di tipo S indica la presenza di un mix di olivine, pirosseni e metalli Fe-Ni; in particolare di silicati: ciò li rende i corpi progenitori più probabili per le meteoriti rocciose.\\
A causa della classificazione di Tholen sono stati introdotti diversi sottogruppi definiti in base alle diverse composizioni e storie evolutive degli asteroidi.\\
Gli asteroidi di tipo S sono situati principalmente nella Main Belt interna e diventano più rari spostandosi verso l'esterno.

\paragraph*{Classe Q}
È una classe di oggetti molto scarna, i pochi corpi di questo tipo sono NEO ma nessuno appartiene alla Main Belt. Un ottimo rappresentante della classe è il NEO (1862) Apollo. Lo spettro è fortemente arrossato attorno a $0.7\,\mu m$ e ha due forti assorbimenti a $1$ e $2\,\mu m$, caratteristiche proprie di un mix di olivine e pirosseni.\\
Questi spettri sono simili a quelli delle condriti ordinarie (CO), che rappresentano circa l'80\% delle meteoriti trovate sulla superficie terrestre. Ciò indicherebbe che gli asteroidi di tipo Q siano i progenitori di queste meteoriti e quindi dovrebbero essere molto numerosi, in contrasto con le osservazioni.

\paragraph*{Classe R}
Questa classe fu inizialmente proposta unicamente per (349) Dembowska. Il suo spettro mostra forti assorbimenti a $1$ e $2\,\mu m$, simili a quelli di Vesta. Dembowska, a differenza di Vesta, presenta caratteristiche allargate verso lunghezze d'onda maggiori, mentre l'assorbimento a $2\,\mu m$ è più stretto e centrato su lunghezze d'onda minori.\\
Le caratteristiche spettrali indicano la presenza di olivine e pirosseni (circa in eguali quantità) e pochi o alcun metalli. 

\begin{figure}
    \centering
    \includegraphics[scale=0.3]{figure/spettro_rv.jpg}
    \caption[Esempio di spettri di tipo V e R.]{Esempio di spettri di tipo V e R. Gli spettri sono normalizzati a $0.55\,\mu m$. Tra parentesi è riportato il range di albedo in percentuale. (Magrin, 2006)}
    \label{spettro_rv}
\end{figure}

\paragraph*{Classe V}
Questa classe fu proposta per (4) Vesta. Il suo spettro è caratterizzato da forti assorbimenti centrati vicino a $1$ e $2\,\mu m$, dovuti ai pirosseni.\\
All'incirca il 6\% degli asteroidi della Main Belt rientra all'interno di questo gruppo. La maggior parte degli asteroidi di tipo V, noti come vestoidi, orbitano nelle vicinanze di Vesta, il che suggerisce che siano dei suoi frammenti generatisi in seguito a impatti con altri corpi. Alcuni NEA sono risultati essere vestoidi: ciò dimostra l'efficacia dei meccanismi di trasporto degli asteroidi dalla Main Belt alle regioni del Sistema Solare interno.

\paragraph*{Classe C}
In letteratura ci si riferisce al gruppo C come un grande gruppo di oggetti scuri e primitivi. Contiene infatti circa il 75\% degli asteroidi noti. Questi oggetti dominano nelle regioni esterne della Main Belt, dove compongono l'80\% di tutta la popolazione, ma diventano più rari spostandosi verso l'interno.\\
Gli asteroidi di tipo C non presentano alcuna caratteristica attorno a $0.4\,\mu m$ ma hanno diverse intensità di assorbimento nell'UV. All'interno di questa classe sono state introdotte le sottoclassi C, B, F, G e P.\\
La classe P non ha assorbimenti nell'UV, la classe F ne ha uno debole, B e C leggermente più intenso e la classe G ha l'assorbimento UV più intenso di tutto il gruppo C. La classe B ha uno spettro da piatto e leggermente blu mentre la classe C da piatto a leggermente rosso. Bell et al. (1989) ha proposto un modello per interpretare le diversità nel gruppo C: le classi B, F e G sarebbero il risultato di un'alterazione metamorfica di un progenitore di classe C. La classe P, che si trova principalmente nella parte esterna della Main Belt, si pensa sia costituita da materiale primitivo contenente una gran quantità di materiale organico, e potrebbe essere una classe di transizione tra C e gli oggetti molto primitivi della classe D.\\
Circa due terzi degli oggetti del gruppo C appaiono idratati, dall'analisi dello spettro attorno a $3\,\mu m$ e nella regione del visibile, probabilmente perché sono stati scaldati al punto che acqua liquida è entrata in contatto con il materiale circostante.

\paragraph*{Classe D}
Gli spettri degli asteroidi di classe D sono generalmente senza caratteristiche particolari, con spettri tendenti leggermente verso il rosso attorno a $0.55\,\mu m$ e spettri molto rossi dopo $0.55\,\mu m$. Hanno, inoltre, albedo molto bassi ($\leq 0.05$).\\
Sono situati nella Main Belt esterna e tra i Troiani di Giove, di cui costituiscono il 60\% della popolazione totale. È stato ipotizzato che questi asteroidi siano composti da una complessa miscela di materiale organico e di regolite\footnote{La regolite è uno strato di frammenti rocciosi incoerenti che forma il terreno superficiale dei vari oggetti nel Sistema Solare.} e che provengano dalla Kuiper Belt.

\begin{figure}
    \centering
    \includegraphics[scale=0.3]{figure/spettro_bgc.jpg}
    \caption[Esempio di spettri di tipo B, G e C.]{Esempio di spettri di tipo B, G e C. Gli spettri sono normalizzati a $0.55\,\mu m$. La scala verticale è ampliata per rendere apprezzabili le differenze tra le classi. Tra parentesi è riportato il range di albedo in percentuale. (Magrin, 2006)}
    \label{spettro_bgc}
\end{figure}

\begin{figure}
    \centering
    \includegraphics[scale=0.3]{figure/spettro_d.jpg}
    \caption[Esempio di spettro di tipo D.]{Esempio di spettro di tipo D. Lo spettro è normalizzato a $0.55\,\mu m$. Tra parentesi è riportato il range di albedo in percentuale. (Magrin, 2006)}
    \label{spettro_d}
\end{figure}

\paragraph*{Classe E}
GLi asteroidi di classe E hanno uno spettro senza caratteristiche, molto simile a quello della classe M. Si differenziano da questi ultimi per via della loro elevata albedo, superiore a $0.3$. La composizione superficiale dovrebbe essere costituita da silicati poveri di ferro, come enstatite e feldspato.\\
Il gruppo degli Hungaria contiene la metà degli asteroidi di classe E noti. Si crede che alcuni NEO di questo gruppo siano i corpi progenitori delle aubriti, una sottoclasse delle meteoriti acondriti.

\begin{figure}[h]
    \centering
    \includegraphics[scale=0.3]{figure/spettro_em.jpg}
    \caption[Esempio di spettri di tipo E e M.]{Esempio di spettri di tipo E e M. Gli spettri sono normalizzati a $0.55\,\mu m$. Tra parentesi è riportato il range di albedo in percentuale. (Magrin, 2006)}
    \label{spettro_em}
\end{figure}

\paragraph*{Classe M}
Si tratta del terzo gruppo più numeroso. Gli spettri sono caratterizzati da un leggero arrossamento e senza particolari caratteristiche. Gli asteroidi di questa classe hanno albedo comprese tra $0.1$ e $0.2$. I dati suggeriscono che la composizione è principalmente ferro e nichel, suggerendo che questi oggetti siano i nuclei metallici differenziati di oggetti più grandi andati distrutti. Si ritengono essere quindi i progenitori delle meteoriti metalliche.

\paragraph*{Classe T}
Gli asteroidi di classe T sono molto rari e hanno albedo molto basse ($\leq 0.1$) con spettri leggermente rossi vero $0.85\,\mu m$. Si ritiene che siano composti da materiale carbonaceo fortemente
alterato (o termicamente o dall'azione dell'acqua).

\begin{figure}
    \centering
    \includegraphics[scale=0.3]{figure/spettro_t.jpg}
    \caption[Esempio di spettro di tipo T.]{Esempio di spettro di tipo T. Lo spettro è normalizzato a $0.55\,\mu m$. Tra parentesi è riportato il range di albedo in percentuale. (Magrin, 2006)}
    \label{spettro_t}
\end{figure}

\subsection{Tassonomia di Bus}
I dati della SMASSII\footnote{\href{http://smass.mit.edu/smass.html}{http://smass.mit.edu/smass.html}} (Small Main-Belt Asteroid Spectroscopic Survey) hanno fornito delle nuove base oer una tassonomia. Bus \& Binzel (2002b) and Bus \& Binzel (2002a) hanno costruito il loro sistema di classificazione basandosi sulle tassonomie pre-esistenti. In particolare hanno definito tre gruppi principali: S-, C- e X-complex, corrispondenti alle definizioni classiche dei tipi S, C e X. Di conseguenza sono state definite 26 classi sulla base della presenza o assenza di alcune caratteristiche spettrali. Di queste 26 classi, 12 hanno una designazione a una lettera: A, B, C, D, K, O, Q, R, S, T, V e X. Una nuova classe, la L, è stata introdotta per descrivere gli oggetti con una pendenza molto ripida nell'UV, vicino a $0.75\,\mu m$, ma con uno spettro piatto oltre i $0.75\,\mu m$. Asteroidi con caratteristiche intermedie sono associati a classi multi-lettera: Cb, Cg, Cgh, Ch, Ld, Sa, Sk, Sl, Sq, Sr, Xc, Xe, e Xk. I membri delle classi Cgh- e Ch- hanno spettri contenenti un assorbimento a $0.7\,\mu m$ legato all'idratazione.\\
Un nuovo asteroide può quindi venire facilmente classificato all’interno di una di queste classi solamente sulla base delle sue caratteristiche spettrali. Di seguito un riassunto delle chiavi tassonomiche della classificazione di Bus-DeMeo.

\begin{figure}[h]
    \centering
    \includegraphics[scale=0.44]{figure/spettro_bus.jpg}
    \caption[Spettri dei tipi tassonomici introdotti da Bus e Binzel.]{Spettri dei tipi tassonomici introdotti da Bus e Binzel. (DeMeo et al., 2009)}
    \label{spettro_bus}
\end{figure}

\section{Space weathering}
Lo space weathering (SW) è un termine che indica dei processi riguardanti i corpi senza atmosfera nel Sistema Solare. Questi processi includono irradiazione da ioni solari (vento solare, flares, espulsioni di particelle energetiche), raggi cosmici galattici, elettroni, raggi UV e X e bombardamenti da parte di micrometeoriti. Le alterazioni superficiali variano non solo in funzione della composizione superficiale ma anche della zona di Sistema Solare in cui si trovano gli oggetti affetti da SW.\\
La Luna è l'oggetto senza atmosfera più studiato e di cui si conoscono meglio gli effetti di SW (Gold, 1955; Hapke, 2001). Il principale responsabile è ferro metallico in nanofase ($npFe^0$)(?) che è prodotto dal bombardamento del vento solare (Keller and McKay, 1997) e il suo effetto è quelli di rendere più arrossati gli spettri di riflettanza e scurire nella regione del visibile-vicino infrarosso (VIS-NIR).\\
Per quanto riguarda gli asteroidi, lo SW è sempre stato tenuto conto per cercare di spiegare possibili corrispondenze (o discrepanze) tra le classi tassonomiche e le meteoriti. Il problema più studiato è quello che concerne la possibilità che gli asteroidi di tipo S (S-type) siano i corpi progenitori delle condriti ordinarie (OC). Numerosi studi si sono concentrati sul confronto della mineralogia degli S-type con le OC, giungendo alla conclusione che potrebbero esserci dei bias nelle collezioni di meteoriti (Meibom and Clark, 1999). Studi successivi hanno dimostrato che esiste un trend nell'arrossamento nel VIS-NIR comparando le OC con asteroidi S-type dalla Main Belt fino ai NEA (Binzel et al., 2004). Ciò suggerisce che, in generale, lo SW lunare è valido anche per gli asteroidi.\\
Alcuni processi di SW possono essere simulati in laboratorio usando diversi approcci e tecniche analitiche (Bennet et al., 2013). I modelli spettrali degli asteroidi sono stati sviluppati utilizzando questi risultati di laboratorio; il risultato principale è che lo SW non modifica il centro della banda e le relative aree di assorbimento dei silicati nei VIS-NIR. Si conosce poco della risposta allo SW di altre classi di asteroidi. Per esempio, i trend di SW diventano più complessi e controversi quando i prendono in considerazione asteroidi più scuri (C-complex). Sono stati osservati dei trend di leggero arrossamento da alcuni (Lazzarin et al, 2006), di spostamento verso il blu da altri (Vernazza et al, 2013).

\subsection{Il caso di (25143) Itokawa}

\subsubsection{Evidenze di SW da osservazioni in situ}
Prima delle osservazioni svolte dalla sonda Hayabusa (2003-2010), osservazioni da Terra nel visibile e infrarosso hanno mostrato che Itokawa è un asteroide di tipo S principalmente composto da olivine e pirosseni, con bande di assorbimento a $1$ e $2\,\mu m$ (Binzel et al., 2001).\\
La sonda Hayabusa ha osservato delle variazioni spettrali che possono essere spiegate con il modello di SW di Hapke (Hapke, 2001), in cui la quantità di $npFe^0$ controlla l'arrossamento spettrale. Analisi recenti dei dati in NIR, riguardo a regioni più blu e "immature" (con pendenze moderate tra $0.43$ e $0.70\,\mu m$), mostrano la banda apparente a $1\,\mu m$ più profonda, confermando i risultati precedenti. Gli spettri del terreno più irregolare variano considerevolmente nel grado di SW, mentre i terreni più regolari sono, in media, debolmente affetti (Ishiguro et al., 2004).\\
I terreni irregolari su Itokawa contengono sia aree scure (e rosse) che aree brillanti (e blu). Le regioni più scure hanno una quantità di massi maggiore, mentre i massi più grandi (qualche decina di metri) appaiono più brillanti dei massi più piccoli.\\
Le caratteristiche osservate nelle zone più brillanti possono essere spiegate se il terreno di Itokawa fosse composto da brecce litificate\footnote{In geologia, processo che conduce alla trasformazione di un sedimento sciolto in roccia sedimentaria.}, poiché le brecce litificate sono caratterizzate da un'eterogeneità mineralogica e petrologica su varie scale, il che modificherebbe il grado di SW.
Inoltre, non ci sono variazioni significanti di colore tra i vari massi, ma quelli più piccoli mostrano delle variazioni di albedo.
\subsubsection{Analisi dei campioni}
La missione Hayabusa ha riportato sulla Terra, il 23 giugno 2010, più di 2000 particelle raccolte dalla superficie dell'asteroide. Le analisi svolte su questi campioni confermano che Itokawa ha una composizione simile alle LL-condriti.\\
Le prime analisi al microscopio hanno riscontrato che metà delle particelle aveva subito alterazioni superficiali (Noguchi et al., 2011). Studi successivi hanno trovato che anche le particelle che sembravano inalterate contenevano delle impurità dovute al ridepositarsi di vapori. In alcuni dei campioni analizzati si sono trovate delle vescicole, il che è probabilmente dovuto all'intrappolamento di He (Noguchi et al., 2014).

Diversi studi hanno mostrato che sebbene lo SW altera gli spettri di riflettanza dei minerali, il cambiamento principale è l'alterazione del continuo spettrale, poiché la posizione delle bande di assorbimento e le relative aree degli assorbimenti dei silicati in VIS-NIR non vengono modificate in modo significativo.\\
La curva del continuo può essere parametrizzata da un coefficiente e può essere legata al tempo di esposizione dello SW. Questo approccio è stato usato per rimuovere gli effetti dello SW dagli spettri di asteroidi S-type, così da poter analizzare la composizione superficiale con più facilità.

\subsection{Stime del tasso di SW sugli asteroidi}
Uno dei vantaggi principale degli studi di laboratorio è che essi possono stabilire quale trend spettrale è dovuto a SW. Questi risultati possono essere usati per determinare se lo SW può causare le variazioni osservate tra gli analoghi terrestri (le meteoriti) e gli asteroidi o le differenze tra asteroidi con composizione simile. I cambiamenti spettrali osservati in laboratorio sono stati usati anche per stimare i tempi scala con cui queste variazioni potrebbero avvenire nell'ambiente spaziale. Per esempio, i tempi scala suggeriscono che gli effetti dovuti al bombardamento di ioni saturano tra $10^4$ e $10^6$ anni, mentre quelli dovuti al bombardamento di micrometeoriti sono nell'ordine di $10^8$ anni (Loeffler et al., 2009, Brunetto et al., 2006b).\\
Nonostante lo SW indotto dagli ioni è un processo abbastanza efficiente, ci sono molti asteroidi che non sono completamente alterati, anche se sono più vecchi dei tempi scala sopracitati. Questa apparente discrepanza indica che qualche processo di ringiovamento è in grado di ricostituire la superficie degli asteroidi, abbassandone il tasso di SW. Tra questi effetti, i più probabili sono piccoli impatti non distruttivi e incontri planetari (Shestopalov et al., 2013).

Gli effetti di SW possono essere derivati anche dallo studio dinamico delle famiglie di asteroidi. La distruzione del progenitore è un processo violento che dovrebbe eliminare la superficie preesistente (alterata o meno) ed esporre nuovo materiale. Questo dovrebbe essere principalmente inalterato, ma sarà alterato dallo SW in tempi successivi. Idealmente, quindi, le famiglie di asteroidi potrebbero essere usate per misurare empiricamente il tasso di SW (invitiamo a consultare Brunetto et al., 2015 per un approfondimento del metodo).

Nonostante siano stati trovati pochi oggetti con spettri inalterati simili alle OC nella Main Belt, esse sono piuttosto comuni tra i piccoli NEA; ciò potrebbe essere dovuto a un particolare processo che interessa i NEA.\\
Usando la tassonomia standard, con cui gli spettri inalterati simili alle OC sono indicati con la lettera Q, si nota che non c'è nessun asteroide di tipo Q tra i NEA con diametro maggiore di $5\,km$. La comparsa degli asteroidi di tipo Q era stata attribuita ai corti tempi di vita collisionali dei piccoli asteroidi, ma è poco probabile poiché i tempi scala collisionali di asteroidi, con dimensione nell'ordine del chilometro, sono presumibilmente $>10^6$ anni, mentre i tempi scala dello SW sono probabilmente minori. Questo problema potrebbe invece essere collegato alla ritenzione della regolite, perché i piccoli asteroidi hanno una gravità più debole e gli strati di regolite alterata vengono rimossi più facilmente da micro-impatti.\\
Recentemente Binzel et al. (2010) ha trovato evidenze che gli incontri planetari dei NEA con la Terra possono rinnovare le superfici degli asteroidi.

L'ultimo processo di SW che tratteremo è quello dell'accrezione e perdita di massa da parte di processi di "rimescolamento" degli starti superficiali che possono esporre materiale "fresco" e/o aggiungere materiale esogeno, precedentemente processato in differenti ambienti.\\
Ci sono due modi per accrescere massa: (1) polvere e meteoroidi e (2) particelle solari energetiche. Il flusso di polvere non è noto, ma si possono ricavare delle stime dul flusso di micrometeoriti basandosi sui modelli dinamici della nube zodiacale. Questo materiale è esso stesso un agente alterante che influenza le proprietà locali della regolite e gli spettri di riflettanza delle regioni dove si è depositato.\\
L'ultimo aspetto da menzionare è la possibilità di perdite di massa indotte dal flusso incidente di ioni e meteoriti. Questo aspetto può essere legato alla ricerca di esosfere attorno ad asteroidi.

\chapter{La missione DART}\label{ch:ch_3}

La missione Double Asteroid Redirection Test (DART) è un dimostratore tecnologico riguardante la difesa planetaria, lanciato e gestito dal Planetary Defense Coordination Office (PDCO) di NASA, in collaborazione con il Johns Hopkins University Applied Physics Laboratory (APL) e l'Agenzia Spaziale Italiana (ASI).

\begin{figure}[!h]
    \centering
    \includegraphics[scale=0.5]{figure/DART_mission_patch.png}
    \caption[Logo della missione DART]{Logo della missione DART. (NASA/JHU-APL)}
    \label{fig:logo_DART}
\end{figure}

L'obiettivo principale della missione è quello di dimostrare l'efficacia dello "\textit{impattatore cinetico}" come metodo per la difesa planetaria dai PHA. Il target scelto per testare la tecnologia è il sistema binario di (65803) Didymos, in particolare la sua luna (65803) Didymos I Dimorphos (d'ora in avanti chiamata semplicemente Dimorphos). Il concetto alla base dell'impattatore cinetico è quello di speronare un asteroide con un oggetto massivo con una velocità sufficientemente alta da cambiare l'orbita del corpo. (Rivskin et al., 2021)

\section{La difesa planetaria}
Come si è visto nella \cref{sec:pha} la regione di spazio attorno alla Terra è densamente popolata di asteroidi potenzialmente pericolosi.\\
Negli anni è maturata sempre più la consapevolezza scientifica che questi oggetti rocciosi possono causare enormi danni al nostro pianeta in caso di collisione, ma al giorno d'oggi non esiste nessun sistema in funzione per la difesa planetaria.

Per questi motivi la comunità scientifica ha iniziato a teorizzare e progettare diversi sistemi anti-collisione con asteroidi (Asteroid Impact Avoidance, AIA), che comprendono tutti quei metodi con cui sarebbe possibile deviare un asteroide in rotta di collisione con la Terra e che provocherebbe eventi distruttivi.\\
Un impatto da parte di un asteroide sufficientemente grande, a seconda del luogo d'impatto, potrebbe causare massicci tsunami, tempeste di fuoco o un "inverno da impatto", dovuto al blocco della luce solare da parte di grandi quantità di polvere e altri detriti immessi nella stratosfera. Si ritiene che una collisione avvenuta 66 milioni di anni fa tra la Terra e un oggetto di 10 chilometri di diametro abbia generato il cratere di Chicxulub e abbia innescato l'estinzione del Cretaceo-Paleogene, che la comunità scientifica ritiene abbia portato all'estinzione di tutti i dinosauri non aviani.\\
Più recentemente, gli eventi di Tunguska del 1908, quando una meteora di circa 65 metri esplose sopra i cieli della Siberia e devastò 2150 km$^2$ di foresta, o di Chelyabinsk del 2013, quando un oggetto di circa 20 metri esplose nei cieli russi causando danni a edifici e infrastrutture, hanno richiamato l'attenzione su tali minacce.

In letteratura sono stati descritti diversi metodi per la difesa planetaria. Tuttavia, nel marzo 2019, gli scienziati hanno riferito che gli asteroidi potrebbero essere molto più difficili da distruggere di quanto si pensasse in precedenza. (El Mir et al., 2019) In aggiunta, un asteroide può anche ricompattarsi grazie alla gravità dopo essere stato distrutto. Nel marzo 2021, al termine della Planetary Defence Conference, gli astronomi e le astronome hanno dichiarato che, per evitare un impatto, potrebbero essere necessari dai 5 ai 10 anni di preparazione preliminare\footnote{\href{https://cneos.jpl.nasa.gov/pd/cs/pdc21/}{https://cneos.jpl.nasa.gov/pd/cs/pdc21/}}; per questo motivo è molto importante avere a disposizione dati orbitali molto precisi.

\subsection{Impattatore cinetico}
Le varie tecniche di prevenzione dalle collisioni presentano diversi compromessi rispetto a parametri quali le prestazioni complessive, i costi, i rischi di missione, le operazioni e le disponibilità tecnologiche. (Canavan, Solomen, 1992) Esistono vari metodi per modificare l'orbita di un asteroide o cometa (Hall, Ross, 1997), che possono essere catalogati in base al tipo di mitigazione (deflessione o frammentazione), la fonte di energia (cinetica, elettromagnetica, gravitazionale, solare/termica o nucleare) e la strategia di approccio (intercettazione o rendezvous (Solem, 1003)).

Per lo scopo di questa tesi ci concentreremo sul metodo dell'impattatore cinetico.\\
Quando un asteroide è ancora lontano dalla Terra, un modo per deflettere la sua orbita è quello di alterare direttamente il suo momento facendo collidere una sonda sulla sua superficie. Delle analisi condotte da NASA nel 2007\footnote{\href{https://web.archive.org/web/20160305101217/http://neo.jpl.nasa.gov/neo/report2007.html}{https://web.archive.org/web/20160305101217/http://neo.jpl.nasa.gov/neo/report2007.html}} mostrarono che "gli impattatori cinetici non nucleari sono l'approccio più maturo e potrebbero essere usati in scenari di deflessione/mitigazione che coinvolgono piccoli oggetti singoli."

Il principio di deviazione, applicato poi dalla sonda DART, si basa sulla legge di conservazione del momento
\begin{equation}
    m_1 \vec{v}_1 + M_2\vec{v}_2 = (m_1+M_2)\vec{v}_3
\end{equation}
con $\vec{v}_1$ e $m_1$ velocità e massa della sonda $\vec{v}_2$ e $M_2$ velocità e massa dell'asteroide e $\vec{v}_3$ velocità finale dopo l'impatto. 
La velocità della sonda rispetto al NEO è nell'ordine dei $10\,km/s$ e il momento è trasferito al NEO, causando un cambio nella sua velocità orbitale che risulta in una leggera deviazione dell'orbita.

\section{Double Asteroid Redirection Test}\label{sec:DART}
DART è una missione NASA che nasce con l'intento di dimostrare la fattibilità del metodo, sopra descritto, dell'impattatore cinetico.\\
NASA ed ESA hanno iniziato a collaborare nel 2015 al progetto AIDA\footnote{\href{https://www.esa.int/Space_Safety/Hera/Asteroid_Impact_Deflection_Assessment_AIDA_collaboration}{https://www.esa.int/Space\_Safety/Hera/Asteroid\_Impact\_Deflection\_Assessment\_AIDA\_collaboration}} (Asteroid Impact and Deflection Assessment), che consiste in due sonde separate ma che lavorano in sinergia (Michel et al., 2018). Dopo varie modifiche al piano della missione si è giunti alla configurazione finale: DART, gestita da NASA, e Hera, di cui parleremo in seguito, gestita da ESA.

Il 24 novembre 2021 la sonda DART è stata lanciata a bordo di un Falcon 9, verso la sua destinazione finale: il sistema di Didymos e Dimorphos.

\subsection{Strumentazione}
Poiché l'obiettivo della sonda era quello di schiantarsi contro un asteroide, essa è stata dotata degli strumenti essenziali per la fase di crociera e di approccio, nonostante ciò aveva una massa di $610\,kg$.

\paragraph*{Camera}
Oltre ai sensori base per la navigazione nello spazio, la sonda ospitava una camera da $20\,cm$ di apertura chiamata DRACO (Didymos Reconnaissance and Asteroid Camera for Optical navigation). DRACO era basata sulla camera LORRI della sonda di NASA New Horizons e ha consentito la navigazione autonoma della sonda fino all'impatto al centro di Dimorphos.\\
La parte ottica della camera era un telescopio Ritchey-Chrétien equipaggiato con un teleobiettivo con campo di vista (FOV) di 0.29° e una lunghezza focale di $2.6208\,m$ (f/12.60). La risoluzione spaziale poco prima dell'impatto è stata di circa $20\,cm/px$.\\
Il sensore utilizzato era un CMOS con un range spettrale di $0.4-1\,\mu m$ (dal visibile al vicino infrarosso).

\begin{figure}
    \centering
    \includegraphics[scale=0.4]{figure/dart_cleanroom.jpg}
    \caption[La sonda DART nella clean room.]{La sonda DART appena installata sul payload adapter. Al centro è possibile vedere il dispenser di LICIACube. (NASA/JHU-APL)}
    \label{fig:dart_cleanroom}
\end{figure}

\paragraph*{LICIACube}
Per osservare l'impatto e l'evoluzione dei detriti (ejecta) da una distanza tale da poter avere un quadro generale del test, ASI e Argotec hanno progettato un cubesat da 6U ($10\,cm\times 20\,cm\times 30\,cm$), LICIACube (Light Italian CubeSat for Imaging of Asteroids), che si è separato da DART l'11 settembre 2022 e ha seguito tutta la missione da una distanza di sicurezza di $\sim 1250\,km$,effettuando un flyby di Dimorphos a $56.7\,km$ 2 minuti e 45 secondi dopo l'impatto della sonda con la luna.

A bordo di LICIACube ci sono due camere ottiche: LEIA (LICIACube Explorer Imaging for Asteroid) e LUKE (LICIACube Unit Key Explorer).

\qquad \small{\textbf{LEIA}} È una camera catadriottica monocromatica con un campo di vista molto stretto (2.06°) e una risoluzione al punto di minima distanza di $1.38\,m/px$.

\qquad \small{\textbf{LUKE}} Si tratta di una camera equipaggiata con un filtro RGB con pattern di Bayern, con un campo largo campo di vista (5°).\\
Durante il flyby Luke ha osservato sia l'emisfero impattato che quello non impattato di Dimorphos con una risoluzione spaziale massima di $\sim 4.3\,m/px$. L'analisi spettrofotometrica dei dati ottenuti verrà usata per indagare la composizione superficiale di Dimorphos; i dati spettrofotometrici permetteranno di indagare l'eterogeneità superficiale e mappare la composizione superficiale dell'oggetto, e anche di cercare per materiali esogeni generati dalla collisione con altri asteroidi durante la storia evolutiva del sistema (Poggiali et al., 2021). I risultati forniti da LUKE, insieme a quelli di LEIA e DRACO, permetteranno di investigare la geologia del sistema di Didymos-Dimorphos (Pajola et al., 2022).

\subsection{Missione primaria}
Dopo 8 mesi di fase di crociera la camera DRACO è riuscita a rilevare Didymos a circa 32 milioni di km e DART ha così potuto iniziare a rifinire la sua traiettoria. 
L'11 settembre 2022, 15 giorni prima dell'impatto, LICIACube è stato rilasciato. 2 minuti e 30 secondi prima dell'impatto l'obiettivo si trovava a $920\,km$. La sonda DART ha impattato sulla luna Dimorphos il 27 settembre 2022 alle 01:14 ora italiana.
L'impatto è avvenuto a $6.6\,km/s$, liberando circa 11 gigajoule di energia.
2 minuti e 45 secondi dopo l'impatto LICIACube ha effettuato il flyby di Dimorphos a $56.7\,km$.

{\footnotesize\textit{ci vanno le immagini ma al momento non funziona il pacchetto}}

\subsection{Campagna di osservazioni}
L'impatto è stato seguito da un gran numero di osservatori terrestri e nello spazio (vedi \cref{fig:obs_dart}).\\
Poiché la strumentazione a bordo della sonda era ridotta all'osso, la missione dipende quasi interamente sui dati raccolti dagli osservatori nei giorni e mesi seguenti l'impatto.
Hubble Space Telescope (HST), James Webb Space Telescope (JWST), l'ATLAS Observatory e l'osservatorio SOAR hanno rilevato il pennacchio di ejecta qualche ora dopo l'impatto, misurandone l'estensione di più di $10000\,km$. Sono state svolte osservazioni anche con il telescopio Copernico (e Galileo) dell'Osservatorio Astrofisico di Asiago e i dati raccolti saranno utilizzati per l'analisi spettroscopica del sistema descritta nei capitoli successivi.\\
Attraverso osservazioni in ottico e in radar si è calcolato che il periodo orbitale di Dimorphos si è accorciato di 32 minuti.

\begin{figure}[h]
    \centering
    \includegraphics[scale=0.113]{figure/Telescopes_observing_DART's_impact.png}
    \caption[Osservatori che hanno osservato l'impatto di DART.]{Lista di osservatori che partecipano alla campagna si osservazioni post impatto. (NASA/JHU-APL)}
    \label{fig:obs_dart}
\end{figure}

\section{Missioni future}
Come già anticipato nella \cref{sec:DART}, la missione DART verrà seguita dalla missione Hera\footnote{\href{https://www.heramission.space/}{https://www.heramission.space/}}, in sviluppo nei centri ESA, che dovrebbe essere lanciata nell'ottobre 2024 e dovrebbe arrivare in orbita attorno al sistema nel dicembre 2026. L'obiettivo principale della missione sarà quello di valutare l'efficacia del metodo dell'impattatore cinetico sul sistema Didymos-Dimorphos. Per soddisfare questo obiettivo Hera dovrà determinare: 
\begin{itemize}
    \item quanto il trasferimento di momento dipende dalla densità, dalla porosità e dalla caratteristiche superficiali e interne dell'asteroide;
    \item in che percentuale l'energia cinetica è trasferita nella frammentazione e nella riorganizzazione dell'asteroide e quanta è trasformata in energia cinetica dei frammenti espulsi dall'impatto.
\end{itemize}
Hera ha anche altri obiettivi scientifici. Infatti, non dovendo schiantarsi contro l'asteroide, sarà equipaggiata con vari strumenti scientifici:
\begin{itemize}
    \item \textbf{Hyperscout-H}: una camera iperspettrale per fornire immagini tra $665$ e $975\,nm$, in 25 diverse bande spettrali.
    \item \textbf{Planetary Altimeter (PALT)}: un micro-altimetro planetario con tecnologia LIDAR. L'accuratezza di misurazione in altezza è di 0.5 metri.
    \item \textbf{Thermal Infrared Imager (TIRI)}: una camera termica a infrarossi fornita dall'Agenzia Spaziale Giapponese (JAXA). Il range spettrale è tra $7$ e $14\,\mu m$ ed è equipaggiata con 6 filtri. La risoluzione spaziale è di 2.3 metri a 10 chilometri di altitudine.
    \item \textbf{X-Band Radio Science (X-DST)}: strumento radio che misurerà l'effetto Doppler dovuto ai disturbi radio e permetterà di ricavare la massa dei due asteroidi, le caratteristiche gravitazionali del sistema binario, le velocità orbitali e le orbite dei due corpi.
    \item Due cubesat: \textbf{Milani} e \textbf{Juventas} il cui scopo sarà quello di completare le informazioni scientifiche raccolte dalla sonda primaria.
\end{itemize}
Questi strumenti saranno necessari per raccogliere informazioni sui due asteroidi: caratteristiche superficiali, porosità e struttura interna. In particolare, Hera sarà la prima missione a mappare completamente la superficie e la struttura interna di un asteroide. L'intera luna Dimorphos verrà mappata globalmente con una risoluzione spettrale di qualche metro e il sito di impatto con una risoluzione di 10 centimetri. Verrà inoltre studiato come si è evoluta la nube di frammenti generatasi dopo l'impatto e le caratteristiche dell'ambiente circostante i due corpi (Michel et al. 2022).

\begin{figure}[!h]
    \centering
    \includegraphics[scale=0.57]{figure/Hera_mission_timeline.png}
    \caption[Timeline della missione Hera.]{Timeline della missione Hera. (ESA)}
    \label{fig:hera_timeline}
\end{figure}
%%vediamo se lasciarla perché non si vedono le scritte. al massimo si sostituisce con un render della sonda.

\chapter{Il sistema di (65803) Didymos}\label{ch:ch_4}
Il sistema di (65803) Didymos è un sistema binario di Near Earth Asteroids.\\
Didymos (parola greca che significa "gemello", data la sua natura binaria), in origine chiamato 1996GT, è stato scoperto l'11 aprile 1996 dalla survey Spacewatch, eseguita dal Kitt Peak Observatory in Arizona. Durante il suo passaggio ravvicinato con la Terra nel novembre del 2003, è stato scoperto il suo satellite naturale Dimorphos (dal greco "che ha due forme", visto che la missione DART ne ha modificato il suo aspetto). Le osservazioni fotometriche sono state svolte dall'Ondrejov Observatory, dal Carbuncle Hill Observatory e dal Steward Observatory; le osservazioni radar sono state fatte dall'osservatorio di Arecibo.\\
Successivamente, tra il 2003 e il 2022, sono state svolte altre osservazioni sia spettrografiche, che fotometriche, che radar per poter caratterizzare al meglio il sistema in vista della missione DART.

\section{Sistemi binari di asteroidi}\label{sec:binary_system}
Negli ultimi decenni il numero di NEA binari è più che quadruplicato e il numero di grandi MBA binari è duplicato. Le attuali evidenze osservative confermano che gli asteroidi binari piccoli ($\lesssim 20\,km$) si sono formati da eventi di fissione rotazionale e stabiliscono che l'effetto YORP è responsabile dell'aumento del tasso di rotazione. I grandi asteroidi ($\gtrsim20\,km$) con piccoli satelliti è più probabile che vengano creati durante grandi eventi collisionali.

\subsection{Introduzione}
I sistemi multipli di asteroidi sono importanti perché rappresentano una frazione considerevole della popolazione asteroidale e perché permettono di investigare molti proprietà, come le proprietà fisiche, la composizione, la struttura interna, e processi, quali i processi di formazione ed evoluzione, che sono spesso difficili da indagare con altri mezzi.\\
L'osservazione di sistemi multipli permette di derivare i valori di massa e densità, con grande accuratezza e precisione, di un gran numero di oggetti. Le misure di densità aiutano a comprendere la composizione e la struttura interna dei corpi minori. I sistemi binari offrono l'opportunità di misurare le proprietà termiche e meccaniche, che sono solitamente poco conosciute.

\subsection{Osservazioni}
\subsubsection{Osservazioni radar dei NEA}
Il radar si è rivelato un ottimo metodo per rilevare i secondari (la componente più piccola di un sistema binario) dei NEA. Nel complesso le osservazioni radar suggeriscono che circa un NEA, più grande di $200\,m$, su sei è un sistema multiplo (Taylor et al., 2012a).\\
Le osservazioni nel regime del radio possono essere usate per rilevare i satelliti di asteroidi grazie all'abilitò di risolvere le componenti di un sistema sia spazialmente (lungo la linea di vista, LoS) che in termini di frequenza (spostamento Doppler della velocità rotazionale e di quella lungo la LoS), col risultato di ottenere una separazione misurabile in due dimensioni tra le componenti.\\
La maggior parte dei sistemi binari di NEA osservati fino a oggi consistono in un primario con una rotazione molto rapida e un secondario più piccolo, nell'ordine di 1/100 la dimensione del primario, la cui rotazione è sincronizzata al periodo orbitale mutuo. La maggioranza dei primari ruota in meno di 2.8 h.

Le forme e i volumi dei corpi, ottenuti dall'inversione delle immagini radar, combinate con la massa del sistema derivata dai moti orbitali, permettono di ricavare la densità del sistema. Basse densità, nell'ordine di $1-2\;g/cm^3$ (Becker et al., 2015, Brozovic et al., 2011), suggeriscono una significante macroporosità interna di almeno il 50\%, implicando una struttura rubble-pile (letteralmente \textit{agglomerato di detriti}). Con queste densità così basse, la rapida rotazione del primario può portare all'accumulo di materiale lungo la fascia equatoriale. La combinazione di rotazione rapida, forma, porosità e struttura rubble-pile hanno implicazioni sul meccanismo di formazione dei sistemi multipli di asteroidi.

\subsubsection{Osservazioni della curva di luce di NEA e MBA}
Una curva fotometrica consiste in una serie temporale di misurazioni della brillanza (è corretto o è meglio luminosità?) totale di un asteroide. La rilevazione di asteroidi binari tramite curva fotometrica si basa sul fatto che le componenti possono oscurare o proiettare un'ombra l'una sull'altra, producendo occultazioni o eclissi rispettivamente. Questo è il metodo principale utilizzato per lo studio del sistema di Didymos post impatto. Le tecniche di osservazione, analisi e modellazione sono state descritte da Pravec et al. (2006), Scheirich and Pravec (2009), Scheirich et al. (2015).

Tra le scoperte principali ottenute con questo metodo è che gli asteroidi binari sono ubiqui. Sono stati trovati tra i NEA, i Mars Crossers, e anche in tutta la Main Belt, sia all'interno di famiglie che non. C'è quindi il sospetto che i sistemi binari siano presenti tra i MBA tanto quanto lo sono tra i NEA. Pare esserci un limite superiore al diametro del primario di circa $13\,km$ (Carry et al. 2015). Un limite inferiore non è ancora stato identificato univocamente.\\
Un'altra scoperta chiave è che i piccoli asteroidi binari hanno un momento angolare vicino al valore critico: questo valore è consistente con la formazione tramite fissione di un corpo progenitore con periodo di rotazione molto rapido e struttura rubble-pile poco coesa (Pravec and Harris, 2007).\\
I poli orbitali degli asteroidi binari nella Main Belt mostrano una distribuzione altamente anisotropica, che si concentra entro 30° dai poli dell'eclittica (Pravec et al., 2012). L'orientazione preferenziale dei poli suggerisce che i corpi progenitori fossero inclinati dall'effetto YORP verso stati di spin asintotici vicino a obliquità di 0 e 180 gradi.\\
Un'altra scoperta rilevante è che pare esserci un limite inferiore al valore della separazione tra le due componenti di un sistema binario $a/D_p=1.5$, corrispondente a un periodo orbitale di 11-12 h.\\
Dalle osservazioni si ricava che i primari hanno una forma pressoché sferica, in gergo top-like (letteralmente \textit{a forma di trottola}).

\subsubsection{Osservazioni spettrali di coppie di asteroidi}
Osservazioni di questo tipo possono essere intraprese quando il secondario ha una grande separazione angolare dal primario, quando il sistema è a eclissi/occultazione, o con l'aiuto di uno spettrografo a campo integrale.\\
Le osservazioni spettrali mostrano che le componenti di un sistema hanno spettri simili. In alcuni casi si è osservata una leggera discrepanza tra le componenti dovuta probabilmente alla differente quantità di SW.

\subsubsection{Imaging diretto di MBA e Troiani}
L'imaging diretto di asteroidi può portare all'identificazione di uno o più satelliti e alla stima dei parametri orbitali. Questa modalità di osservazione rimane impegnativa perché i satelliti sono generalmente molto più piccoli e deboli rispetto ai primari e perché la maggior parte dei satelliti conosciuti orbita a distanze angolari sotto il secondo d'arco. La strumentazione in grado di ottenere i requisiti di contrasto e risoluzione include l'HST e i telescopi terrestri ($>10\,m$) equipaggiati con ottica adattiva.

La frazione di grandi MBA con satelliti è difficile da stimare data la complessa dipendenza dell'osservabilità del satellite dalla separazione angolare e del rapporto di flussi tra primario e secondario. I satelliti conosciuti hanno un rapporto tra le masse con il primario tra $10^{-6}$ e $10^{-2}$. Tutti hanno un periodo orbitale tra 1 e 5.5 giorni. Molte orbite hanno eccentricità vicine a zero, dovute a effetti di smorzamento tramite forze mareali.

\subsection{Formazione dei sistemi di piccoli asteroidi}
Si è visto come gli asteroidi primari di piccoli sistemi binari hanno rotazioni molto rapide. Inoltre quasi tutti i piccoli asteroidi binari conosciuti hanno un elevato momento angolare. Queste caratteristiche non sono consistenti con una formazione in seguito a un sub-catastrofico, a cattura tramite interazione a tre corpi o a cattura dopo un impatto catastrofico. Sono invece indice di una formazione tramite fissione rotazionale. La fissione rotazionale prevede che un corpo progenitore possa essere messo in una rotazione così rapida che le forze centrifughe sovrastano quella gravitazionale, che tiene coeso l'asteroide.

Jacobson e Scheeres (2011b) hanno mostrato che dopo la fissione rotazionale ci sono varie possibili alternative all'evoluzione del sistema. Le loro simulazioni numeriche hanno portato alla stesura del diagramma mostrato in \cref{fig:binaries_evolution}. La distinzione tra alto e basso rapporto di massa $q$ gioca un ruolo importante nel distinguere due percorsi evolutivi. Lungo $q\gtrsim0.2$, entrambi i corpi si sincronizzano tramite forze mareali ed evolvono poi in base all'effetto Binary-YORP.\\
Lungo $q\lesssim0.2$ il sistema binario è non-legato. Durante la fase caotica il sistema può andare incontro a un altro episodio di fissione rotazionale. La materia persa da un corpo può accrescere il primario, contribuendo alla formazione della cresta equatoriale, o può fuggire dal sistema. Se avviene quest'ultimo evento ci sono tre possibili risultati: si forma un asteroide con una forma diversa, una coppia di asteroidi o un sistema binario stabile. Successivamente, in quest'ultimo caso il secondario si sincronizza con il primario a causa delle forze mareali. In seguito il sistema evolve in base a come agiscono l'effetto BYORP e le forze mareali.

\begin{figure}
    \centering
    \includegraphics[scale=0.73]{figure/binaries_evolution.jpg}
    \caption[Diagramma dei possibili percorsi evolutivi di un piccolo asteroide.]{Diagramma dei possibili percorsi evolutivi di un asteroide dopo la fissione rotazionale. Le fasi sottolineate sono stabili in un tempo scala dell'effetto YORP. $q$ è il rapporto tra la massa del secondario e del primario. (Jacobson e Scheeres, 2011b)}
    \label{fig:binaries_evolution}
\end{figure}

\section{Caratteristiche dinamiche}
Il sistema di Didymos, con un afelio di 2.2753 AU e un perielio di 1.0131 AU, fa parte della famiglia Apollo poiché il suo perielio è minore di 1.017 AU, afelio dell'orbita terrestre (vedi \cref{tab:asteroid-classification}).

\begin{figure}[!h]
    \centering
    \includegraphics[scale=0.18]{figure/orbit_viewer.jpg}
    \caption[Vista dell'orbita di (65803) Didymos.]{Vista dell'orbita di (65803) Didymos, in bianco, confrontata con quella di Terra e Marte, rispettivamente in blu e rosso. (NASA/JPL)}
    \label{fig:orbit_view}
\end{figure}

Dati dati radio raccolti nel 2003 e rianalizzati da Naidu et al. (2020), la larghezza di banda del secondario suggerisce un periodo di rotazione di $12.4\pm 3.0h$, valore consistente con una rotazione che è sincronizzata con il periodo orbitale mutuo di 11.9 h. Usando il delay nel segnale e le separazioni Doppler si è ottenuto un valore del semiasse maggiore del sistema di $1190\pm 30\, m$, un'eccentricità $<0.05$ e un periodo orbitale di $11.93\pm 0.01$ h, valori consistenti con i precedenti lavori di Scheirich and Pravec (2009) e Fand and Margot (2012). 

\begin{figure}[!h]
    \centering
    \includegraphics[scale=0.78]{figure/system_orbit.jpg}
    \caption{Geometria del sistema di Didymos. In giallo e arancio sono evidenziati i sitemi di riferimento. (Bottiglieri et al., 2022)}
    \label{fig:system_orbit}
\end{figure}

Il valore osservato della deriva quadratica (quadratic drift) nell'anomalia media è $\Delta M_d=0.15\pm 0.14\; deg\,yr^{-1}$. Esprimendo $\Delta M_d$ in termini del semiasse maggiore dell'orbita mutua $a$, $\Delta M_d=-\frac{3n\dot{a}}{4a}$, la deriva dell'anomalia media implica uno spostamento verso valori minori del semiasse maggiore con un tasso \\$\dot{a}=-0.09 \pm 0.08\;cm\,yr^{-1}$ (Scheirich and Pravec, 2022). 

Due sono i possibili meccanismi in grado di spiegare questa deriva.\\
L'effetto YORP binario (BYORP) è un cambio secolare dell'orbita reciproca di un sistema binario di asteroidi con un satellite sincrono dovuto all'emissione di radiazione termica dalla forma simmetrica di quel satellite. Pravec and Scheirich (2010) hanno previsto che il valore della deriva quadratica di Didymos dovuta all'effetto BYORP sarebbe dovuta essere pari a $2.51\;deg\,yr^{-1}$, valore molto superiore del valore stimato dai dati analizzati da Scheirich and Pravec (2022). Più recentemente, Jacobson and Scheeres (2011) hanno presentato una teoria migliorata che incorpora sia l'effetto BYORP che le forze mareali tra le due componenti. Hanno mostrato che potrebbe esistere un equilibrio stabile a lungo termine tra queste due forze, nel caso in cui l'effetto BYORP sottragga momento angolare all'orbita. Poiché il periodo di rotazione di Didymos è minore del periodo orbitale di Dimorphos, il satellite crea un rigonfiamento mareale sul primario che sottrae energia alla rotazione del primario e la trasferisce al momento angolar orbitale. Queste due forze sono di segno opposto e possono bilanciarsi a vicenda perché dipendono in modo differente dal valore del semiasse maggiore. Queste forse fanno evolvere l'orbita reciproca fino a un valore del semiasse maggiore in cui essa è in equilibrio e quindi non evolve ulteriormente. Il valore riscontrato della deriva dell'anomalia media potrebbe indicare che il sistema sta evolvendo, ed è molto vicino, a questo equilibrio.\\



\section{Caratteristiche fisiche e morfologiche}
Naidu et al. (2020) analizzando i dati radar raccolti con il radiotelescopio di Arecibo nel 2003, e includendo le curve di luce ottenuta da Pravec et al. (2006), ha ottenuto una stima del modello della forma del sistema di Didymos. Il primario, Didymos, è a forma "top-shaped" ("a forma di trottola") con un rigonfiamento equatoriale. Il diametro equivalente è di $780\pm 30\,m$ e le estensioni lungo i tre assi principali sono $832\,m$, $838\,m$, $786\,m$ (con un'incertezza del 6\% sugli assi x e y, e del 10\% sull'asse z). La densità è $2.17\pm 0.35\;g\,cm^{-3}$. I dati radar non contengono una rotazione completa del secondario, Dimorphos, ma mostrano un'estensione visibile di circa $75\,m$, implicando un diametro di $150\pm 30\,m$.

\subsection{Modellizzazione della forma}\label{sec:shape_model}

\begin{figure}[!h]
    \centering
    \includegraphics[scale=1.3]{figure/didymos_241103.jpg}
    \caption[Immagini delay-Doppler ottenute con il telescopio di Arecibo.]{Immagini delay-Doppler ottenute con il telescopio di Arecibo il 24 novembre 2003 tra le 03:15 e le 05:09 UTC. Ogni immagine è la somma di tre pose, durante le quali il primario ha ruotato di 14°. La risoluzione dell'immagine è di $0.1\,\mu s\times 0.3\,Hz$. (Naidu et al., 2020)}
    \label{fig:didymos_241103}
\end{figure}

La \cref{fig:didymos_shape_model} mostra la vista dei principali assi del modello di "best-fit" finale basato sui dati sopra descritti. Una delle caratteristiche più prominenti è una facet/faccia/sfaccettatura\footnote{Superficie piatta su una figura geometrica.} sul rigonfiamento equatoriale lunga $\sim 350\,m$. Un'altra piccola facet/faccia è presente a circa 120° in longitudine dalla prima. Il resto della superficie appare regolare su scale inferiori a $100\,m$, ma è un fattore probabilmente dovuto alla mancanza di caratteristiche nei dati dovuta al basso rapporto segnale-rumore ($S/N$) e alla bassa risoluzione, e non implica che la superficie sia effettivamente regolare su scale decametriche.\\
La maggior parte dei punti delle curve di luce corrispondono in modo ottimale al modello della forma, ma ci sono delle eccezioni (\textit{outliers}). Queste potrebbero essere dovute a diversi fattori quali una topografia su scala minore che non è visibile nelle immagini radar, uno scattering ottico non ottimale sulla superficie e le assunzioni fatte nel definire i modelli di scattering ottico e radar.

\begin{figure}
    \centering
    \includegraphics[scale=0.78]{figure/didymos_shape_model.jpg}
    \caption[Vista degli assi principali del modello della forma del primario.]{Vista degli assi principali del modello della forma del primario. Il modello ha 1000 vertici e 1996 facets triangolari. La risoluzione efficace è $\sim 50\,m$. Le regioni in giallo indicano regioni in cui l'angolo di incidenza e di scattering è $>60^\circ$. (Naidu et al., 2020)}
    \label{fig:didymos_shape_model}
\end{figure}

\subsection{Formazione del crinale equatoriale}
Per lo studio della formazione del rigonfiamento equatoriale sono stati presi in considerazione due modelli con tassi di rotazione diversi (\cref{fig:failure_models}): uno con il periodo di rotazione attuale di 2.26 h, l'altro con un periodo più lungo, di 3.5 h, che è il caso in cui si trovava Didymos nel passato.\\
Nel caso del periodo di 2.26 h si ottiene che l'asteroide debba essere andato incontro a un cedimento strutturale interno. Questo ha comportato una deformazione verso l'esterno del piano equatoriale e una deformazione verticale in altre regioni, modalità di cedimento consistente con altri asteroidi top-shaped con rotazioni rapide (Hirabayashi and Scheeres, 2019).\\
Diversamente, il caso del periodo di rotazione di 3.5 h mostra una modalità di cedimento differente. Hirabayashi (2015) ha previsto che se un oggetto sferoidale che ruota lentamente ha una coesione vicino a zero, allora le condizioni superficiali diventano più sensibili al cedimento rispetto a quelle interne. Il modello mostra, quindi, zone di cedimento locale sparse per tutta la superficie ma che tendono ad collocarsi a basse latitudini, dove la pendenza del campo gravitazionale è ripida. I vettori di deformazione puntano principalmente verso l'interno.

\begin{figure}
    \centering
    \includegraphics[scale=0.6]{figure/failure_models.jpg}
    \caption[Soluzioni dei modelli di cedimento basati sull'analisi di Didymos.]{Soluzioni dei modelli di cedimento basati sull'analisi di Didymos. Le figure $a$ e $b$ mostrano la soluzione nel caso di un periodo di rotazione di 2.26 h; $c$ e $d$ il caso con un periodo di 3.5 h. Le frecce sono indicano i vettori di deformazione. (Naidu et al., 2020)}
    \label{fig:failure_models}
\end{figure}

Una rotazione rapida contribuisce alla formazione di un asteroide top-shaped e lavori precedenti (Walsh et al., 2008, 2012) hanno ipotizzato che sia frane che deformazioni interne possono portare alla formazione di un crinale equatoriale. Al momento non è possibile identificare quale sia stata la causa della formazione del rigonfiamento, ma lavori futuri che analizzeranno i dati raccolti da DART, LICIACube e Hera potranno fare chiarezza su questo aspetto. In particolare, immagini dettagliate della morfologia superficiale di Didymos potranno essere usate per determinare come la deformazione avviene in un ambiente di microgravità e fornire vincoli alla struttura interna dell'asteroide.

\subsection{Asteroidi rubble-pile}
I modelli ottenuti utilizzando i dati delle osservazioni descritte nella \cref{sec:binary_system} hanno suggerito che la maggioranza degli asteroidi tra $\sim 0.2$ e $10\,km$ sono rubble-pile. In consistenza con queste aspettative, gli asteroidi Itokawa ($0.3\,km$), Bennu ($0.5\,km$) e Ryugu ($0.9\,km$) sembrano essere rubble-pile. Per di più, il confronto dei massi più grandi con i crateri più larghi ha rivelato che la popolazione di rocce grandi è stata ereditata dalla distruzione catastrofica di un corpo progenitore (DellaGiustina et al., 2019).\\
Una recente analisi dei campioni recuperati dalla superficie di Ryugu ha rilevato che la densità del materiale di cui sono fatti i campioni è simile alla densità globale dell'asteroide, indicando un'alta microporosità e una bassa macroporosità (Pilorget et al. 2022). Ciò è in contrasto con Itokawa, dove la macroporosità è alta e la microporosità bassa (Tsuchiyama et al. 2014)\\
Questi risultati indicano la possibilità di un forte contrasto tra le caratteristiche della porosità interna di asteroidi appartenenti a diverse classi tassonomiche e ha implicazioni importati per quanto riguarda Didymos e Dimorphos. È anzi probabile che Didymos, con la sua forma top-shaped e la rapida rotazione, sia un oggetto rubble-pile formatosi da una collisione (Zhang et al. 2021). Inoltre, analizzando qualitativamente le immagini, immortalate da LICIACube dello scontro tra DART e Dimorphos, si potrebbe dedurre che anche il secondario sia di tipo rubble-pile. Saranno però necessari studi approfonditi che verranno pubblicati nel prossimo futuro.

\subsection{Indizi sulla formazione ed evoluzione del sistema}
È importante sottolineare che le condizioni geologiche osservare su Didymos e Dimorphos potrebbero correlare direttamente con i meccanismi di formazione ed evoluzione del sistema, quali: forze mareali, impatto, effetto rotazionale del primario, luna originale di un corpo progenitore.\\
Prendendo in considerazione gli attuali dati radar (Naidu et al., 2020) e curve di luce (Pravec et al., 2016), è possibile escludere le forze mareali come meccanismo rilevante per la formazione ed evoluzione di Dimorphos. Infatti, le forze mareali fanno sì che il corpo precursore si deformi in un corpo allungato. Al contrario, secondo i modelli descritti nella \cref{sec:shape_model}, Didymos è pressoché assisimmetrico, in contraddizione con la forma attesa dovuta alle forze mareali. Le curve di luce più recenti suggeriscono un oggetto leggermente più piccolo, più sferico e più denso (Pravec and Scheirich 2022).\\
Per quanto riguarda lo scenario dell'impatto, l'esistenza di grandi crateri su Didymos potrebbero essere un potenziale indicatore. Nell'eventualità che Didymos si riveli essere monolitico, questo scenario è escluso perché per generare un corpo delle dimensioni di Dimorphos servirebbe un impatto che genererebbe un cratere di 10 km di diametro, molto più grande rispetto alla dimensione del primario. Nel caso di un impatto subcatastrofico, invece, potrebbe non esserci un'evidenza univoca di crateri. Una grande quantità di materia potrebbe essere espulsa da Didymos a basse velocità, contribuendo alla formazione ed evoluzione del satellite. In questo caso Didymos presenterebbe numerose caratteristiche topografiche asimmetriche; sarebbe possibile, quindi, trovare materiale esogeno sulla superficie dell'asteroide.\\
La rapida rotazione, di 2.26 h, a bassa latitudini comporta una pendenza superficiale che eccede il valore tipico dell'angolo di frizione di materiale granulare ($\sim 35^{\circ}$; Lambe and Whitman, 1969). Il periodo di rotazione potrebbe essere abbastanza breve da indurre un rimodellamento su varie scale di lunghezza: da movimenti di massa in zone confinate della superficie a una riconfigurazione globale. Lo scenario dell'effetto rotazionale deriva dalle espulsioni di massa dovute al breve periodo di rotazione. In questo caso, in base alla presenza di coesione, la formazione di Dimorphos potrebbe essere diversa. Se c'è coesione, una roccia monolitica potrebbe aver lasciato Didymos. Non è ancora chiaro quale potrebbe essere il lascito di questo evento, dato che sui piccoli asteroidi i massi più grandi sono ben sepolti e non si muovono facilmente (Daly et al., 2020a). Se le particelle non sono coese, la perdita di materiale dalla superficie e la sua successiva riaccumulazione potrebbe portare alla formazione di un satellite (Walsh et al., 2012). Per lo scenario senza coesione, i due asteroidi avrebbero la stessa composizione, ma la presenza di regolite fine ($<1\,cm$) potrebbe essere differente. La pressione di radiazione solare potrebbe aver avuto abbastanza tempo per spazzare via le particelle più piccole. L'evoluzione orbitale degli ejecta, da Didymos a Dimorphos, potrebbe portare a essere più giovane la faccia di Dimorphos rivolta verso Didymos, se l'orbita reciproca è effettivamente in uno stato di rilassamento dinamico (Agrusa et al., 2021).\\
La geologia superficiale potrebbe anche dipendere dall'evoluzione delle orbite tra Didymos e Dimorphos. Per esempio, l'accelerazione dell'anomalia media di Dimorphos è stata calcolata essere bassa, il che fa propendere per due possibili scenari. Per quanto riguarda il primo, se questa bassa accelerazione dell'anomalia media è dovuta al moto caotico di Dimorphos, dato l'azione limitata dell'effetto BYORP (radiazione termica che causa la perturbazione delle dinamiche reciproche), la superficie di Dimorphos potrebbe essere rivolta caoticamente verso Didymos, consentendo accumuli di materiale di Didymos un po' ovunque. In questo scenario, il riscontro geologico sarebbe trovare materiale di età simile su tutta la superficie di Dimorphos. Il secondo scenario è che, nonostante l'effetto BYORP sia significativo questo venga cancellato dalle forze mareali del primario sul secondario. In questo caso, una faccia di Dimorphos sarebbe sempre rivolta verso il primario, portando a una dicotomia nei colori delle due facce. 

\section{Didymos come obiettivo della missione DART}


















\end{document}