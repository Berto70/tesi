\documentclass[a4paper,11pt,openright]{book}

\usepackage[top=2cm,bottom=2cm,left=2cm,right=2cm]{geometry}
\usepackage[pdftex]{graphicx} %per poter inserire le figure
\usepackage{amssymb,amsmath,amsthm,amsfonts}
\usepackage{xspace}
\usepackage{tabularx}
\usepackage{indentfirst}
\usepackage{subfigure}
\usepackage[small]{caption}
\usepackage{eucal}
\usepackage{eso-pic}
\usepackage{url}
\usepackage{booktabs}
\usepackage{afterpage}
\usepackage{parskip}
\usepackage{listings}
\usepackage{fancyhdr}
\usepackage{textcomp}
\usepackage{cite}
\usepackage{multirow}
\usepackage[utf8]{inputenc}   %per riuscire a scrivere gli accenti
\usepackage[italian, english]{babel}   %per riuscire a scrivere gli accenti
\usepackage{setspace}
%\usepackage[swapnames]{frontespizio}
%\pagestyle{fancy} 

\newenvironment{abstract}%
{\cleardoublepage%
\thispagestyle{empty}%
\null \vfill
\begin{center}%
\Huge \bfseries \abstractname 
\end{center}}%
{\vfill\null}

%%%%%%%%%%%%%%%%%%%%%%%%%%%%%%%%%%%%%%%%%%%%%

\begin{document}
\frontmatter

\begin{titlepage}
    \vspace{5mm}
    \begin{figure}[hbtp]
        \centering
        \includegraphics[scale=.09]{figure/UNIPD.png}
    \end{figure}
    \vspace{5mm}
    \begin{center}
        {{\huge{\textsc{\bf UNIVERSIT\`A DEGLI STUDI DI PADOVA}}}\\}
        \vspace{5mm}
        {\Large{\bf Dipartimento di Fisica e Astronomia ``Galileo Galilei''}} \\
        \vspace{5mm}
        {\Large{\textsc{\bf Corso di Laurea in Astronomia}}}\\
        \vspace{20mm}
        {\Large{\textsc{\bf Tesi di Laurea}}}\\
        \vspace{20mm}
        \begin{spacing}{3}
            {\LARGE \textbf{Caratterizzazione spettroscopica del sistema di Didymos dopo l'impatto con la sonda DART}}\\
        \end{spacing}
        \vspace{8mm}
    \end{center}
    
    \vspace{20mm}
    \begin{spacing}{2}
        \begin{tabular}{ l  c  c c c  cc c c c c  l }
            {\Large{\bf Relatrice}} &&&&&&&&&&& {\Large{\bf Laureando}}\\
            {\Large{\bf Prof.ssa Monica Lazzarin}} &&&&&&&&&&& {\Large{\bf Gabriele Bertinelli}}\\
            {\Large{\bf Correlatrice}}\\
            {\Large{\bf Prof.ssa Fiorangela La Forgia}}\\
        \end{tabular}
    \end{spacing}
    \vspace{15 mm}    
    \begin{center}
        {\Large{\bf Anno Accademico 2022/2023}}
    \end{center}
\end{titlepage}

\clearpage{\pagestyle{empty}\cleardoublepage}

%%Dedica
\null\vspace{\stretch{1}}
\begin{flushright}
\textit{Alla mia famiglia}
\end{flushright}
\vspace{\stretch{2}}\null
\clearpage{\pagestyle{empty}\cleardoublepage}

%%Citazione
\null\vspace{\stretch{1}}
\begin{flushright}
\textit{Complete the motion if you stumble\\RHCP}
\end{flushright}
\vspace{\stretch{2}}\null
\clearpage{\pagestyle{empty}\cleardoublepage}

%%Sommario
\selectlanguage{italian}
\begin{abstract}
    %\begin{center}
        Qui ci sarà il riassunto della tesi.\\
        16 novembre 2022 è la data dell’inizio di una nuova era di missione spaziali: l’era delle missioni Artemis, che si ripromettono di riportare sulla Luna l’essere umano.

Alle 07:47, ora italiana, il razzo Space Launch System di NASA ha illuminato a giorno il pad 39-B al Kennedy Space Center, dando inizio alla missione Artemis I.
Scopo della missione è testare SLS nel suo complesso, razzo, capsula, sottosistemi, procedure ecc. 
Dopo 26 giorni di missione, l’11 dicembre, la capsula Orion rientrerà nell'Oceano Pacifico, al largo delle coste di San Diego.

Questa mattina, dopo qualche problema nei centri di controllo, il lancio è stato nominale: i booster migliorati, ereditati dalle missioni Shuttle, hanno performato bene; il core centrale di SLS (il cilindro arancione) idem; i 4 pannelli solari del Service Module, costruito dall’ESA (con grande contributo italiano) si sono aperti nel modo corretto e stanno generando la corrente elettrica necessaria.
Attorno alle 09:14 è iniziata la Trans Lunar Injection Maneuver, manovra per spostarsi dall'orbita terrestre a quella lunare, durata 18 minuti (un record per il motore RL10). Tutto è andato come previsto e Orion è in rotta verso la Luna.

I prossimi step sono la separazione dell'ICPS, il 2o stadio che ha permesso le ultime due manovre, correzioni dell’orbita e il rilascio di 13 cubesat, tra cui ArgoMoon: cubesat costruito da Argotec e ASI, azienda aerospaziale italiana, con l’obiettivo di monitorare da vicino tutte le operazioni che compirà la Orion attorno alla Luna.

Nei prossimi giorni seguiranno aggiornamenti sulla missione.

    %\end{center}
\clearpage{\pagestyle{empty}\cleardoublepage}
\end{abstract}

\tableofcontents

\listoftables

\listoffigures

\clearpage{\pagestyle{empty}\cleardoublepage}

\mainmatter
\chapter{Corpi minori del Sistema Solare}

\section{Proprietà generali}

Dal 1801, quando Giuseppe Piazzi scoprì il primo corpo minore Cerere, categorizzato poi come asteroide, per sottolineare le differenze apparenti rispetto alle comete, si sono scoperti e categorizzati più di 1 milione di asteroidi (https://solarsystem.nasa.gov/asteroids-comets-and-meteors/asteroids/overview/). La maggior parte orbita tra Marte e Giove e forma quella che è chiamata Main Belt (2-3.3 AU).

Esistono numerosi gruppi di asteroidi, a diverse distanze dal Sole. Nella Tabella sono riportati i principali gruppi e famiglie, con le loro classificazioni dinamiche.

In questo capitolo verranno descritte brevemente le principali caratteristiche. La loro definizione è spesso legata alla risonanza con Giove o Nettuno, mentre altri gruppi sono definiti dall'intersezione della loro orbita con quella di un pianeta (i.e. Near Earth Objects e Mars Crossers).

\subsection{Effetto Yarkovsky} %%aggiusta con info tesi balossi
L’effetto Yarkovsky è una forza percepita da un corpo dovuta all’emissione anisotropa di fotoni termici, che trasportano momento. La sua influenza è più significativa per meteoroidi e piccoli asteroidi (da 10 cm a 10 km di diametro).

L’effetto fu scoperto da Ivan Osipovich Yarkovsky, che notò come il riscaldamento diurno di un oggetto rotante nello spazio causasse una piccola forza sull’oggetto che avrebbe avuto effetti a lungo termine sulla sua orbita.

L’effetto è dovuto a due componenti:


\qquad \textit{Effetto diurno}: su un corpo rotante illuminato dal Sole, la superficie è più calda nel pomeriggio e nella prima parte della notte, rispetto alla mattina e a notte tarda. Il risultato è che viene radiato più calore dal lato notturno, rispetto a quello diurno, e, al netto, agisce una forza dovuta alla pressione di radiazione nella direzione opposta alla notte. Per oggetti rotanti in modo progrado, questa forza è nella direzione della loro orbita e il semiasse maggiore, nel lungo periodo, tende ad aumentare. Situazione opposta per gli oggetti che ruotano in modo retrogrado.


\qquad \textit{Effetto stagionale}: per oggetti rotanti, l’effetto stagionale aumenta con l’inclinazione assiale. Esso domina solo se l’effetto diurno è abbastanza piccolo.
L’effetto stagionale potrebbe avvenire per:


    \begin{itemize}
        \item rotazione molto veloce
        \item piccole dimensioni dell’oggetto
        \item inclinazione assiale vicina a 90°
    \end{itemize}


Anche questo effetto ha ripercussioni sul semiasse maggiore, e nell’ordine dei milioni di anni può portare un asteroide dalla Main Belt al Sistema Solare interno.

\paragraph*{Effetto YORP}
L’effetto Yarkovsky-O'Keefe-Radzievskii-Paddack (YORP) è un variazione al 2o ordine dell’effetto Yarkovsky che causa l’aumentare o il diminuire della velocità di rotazione (o spin) di un piccolo corpo, come un asteroide.\\
Nel lungo termine, l'inclinazione e il tasso di rotazione dell'oggetto possono variare in modo casuale, in modo caotico o regolare, a seconda di molti fattori.

\section{Distribuzione e classificazione dinamica}
In questo capitolo verrà decritta la classificazione in gruppi e famiglie.
La classificazione è riassunta nella Tabella.\\
I corpi minori sono divisi in gruppi e famiglie in base alle loro caratteristiche orbitali. 
Groups are relatively loose dynamical associations, whereas families are tighter e sono probabilmente il risultato di una distruzione catastrofica di un asteroide “genitore” (in inglese è tutto così bello “parent asteroid”).

Per la classificazione è utile conoscere le definizioni analitiche di afelio e perielio di un'orbita:
\begin{equation}
    \begin{cases}
        Q=a(1+e) &\text{afelio}\\
        q=a(1-e) &\text{perielio}
    \end{cases}
\end{equation}

dove \textit{a} e \textit{e} sono rispettivamente il semiasse maggiore e l'eccentricità dell'orbita.

\paragraph*{Near Earth Objects}
I Near Earth Objects (NEOs) rappresentano un gruppo eterogeneo di asteroidi (NEAs) e nuclei cometari estinti (NECs) che hanno orbite con un perielio minore di quello di Marte: $q<1.3$ AU. Si ritiene che i NEOs siano la principale fonte di meteoriti che arrivano sulla Terra.\\
In base alla relazione tra la loro orbita e quella della Terra sono catalogati in diversi sotto-gruppi: \textit{Atens} e \textit{Apollo} sono Earth Crossers (i.e. intersecano l’orbita della Terra), mentre gli \textit{Amors} si avvicinano alla Terra ma non intersecano mai la sua orbita. Gli Atens hanno un semiasse maggiore più piccolo di quello della Terra ($a<1$ AU) mentre gli Apollo hanno $a>1$ AU.

\paragraph*{Mars Crossers}
Come fa intendere il nome, i \textit{Mars Crossers} sono oggetti che intersecano l’orbita di Marte.\\
È stato mostrato che la popolazione dei Mars Crossers, che è circa 4 volte più grande di quella dei NEOs, è rifornita da risonanze diffusive nella Main Belt (Migliorini et al. 1998; Morbidelli \& Nesvorny 1999; Michel et al. 2000; Bottke et al. 2002), dalla regione chiamata “intermediate-source Mars-crossing region” o IMC.

\paragraph*{Main Belt Asteroids}
Si tratta di asteroidi situati tra le orbite di Marte e Giove. La maggior concentrazione si trova tra 2.0 e 3.3 AU.\\
A causa delle risonanze orbitali dovute all’influenza gravitazionale di Giove, si vengono a creare molti gruppi, sotto-gruppi e famiglie. Di seguito sono riportati i più popolosi.

\qquad \textit{Hungaria}: prendono il nome da 434 Hungaria. Hanno un semiasse maggiore tra 1.78 e 2.06 AU, un’eccentricità minore di 0.18 e un’inclinazione tra 16° e 34°. 

\qquad \textit{Inner Main Belt}: hanno un semiasse maggiore tra 2.06 e 2.50 AU, limiti definiti dalle risonanze di moto medio 4:1 e 3:1.

\qquad \textit{Middle Main Belt}: hanno un semiasse maggiore tra 2.50 e 2.82 AU. I limiti sono definiti dalle risonanze di moto medio 3:1 e 5:2.

\qquad \textit{Outer Main Belt}: hanno un semiasse maggiore tra 2.82 e 3.28 AU. All’interno del gruppo troviamo le famiglie \textit{Koronis}, \textit{Eos} e \textit{Themis}.

\qquad \textit{Cybele}: hanno un semiasse maggiore tra 3.28 e 3.70 AU. Prendono il nome da 65 Cybele.

\qquad \textit{Hilda}: hanno un semiasse maggiore medio tra 3.70 e 4.20 AU. Prendono il nome da 153 Hilda.

\paragraph*{Jupiter Family Comets}
Le comete sono divise in due gruppi: comete a lungo periodo e comete a corto periodo.\\
Le comete a corto periodo hanno un periodo orbitale <20 anni a basse inclinazioni. Hanno anche $2<T_J<3$, con $T_J$ Invariante di Tisserand, espressa da

\begin{equation}
    T_J=\frac{a_J}{a}+2\cos(i)\biggl[\frac{a}{a_J}(1-e)\biggr]^{1/2}
\end{equation}

dove $a$ e $a_J$ sono i semiassi maggiori dell'oggetto e di Giove rispettivamente ed $e$ è l'eccentricità dell'oggetto.\\
Visto che la loro orbita è controllata da Giove sono chiamate Jupiter Family Comets (JFCs). Si pensa che le comete a corto periodo provengano dalla Kuiper Belt, una grande riserva di piccoli corpi ghiacciati che si trova oltre Nettuno. A causa di collisioni e/o perturbazioni gravitazionali alcuni oggetti della Kuiper Belt riescono a fuggire dalla stessa e dirigersi verso il Sistema Solare interno. La successiva interazione con Giove può far sì che questi oggetti assumano parametri orbitali simili a quelli degli asteroidi.

\paragraph*{Centaurs}
I Centaurs sono oggetti che orbitano tra Giove e Nettuno ($5.4<a<30$ AU), ma non hanno ancora una definizione dinamica univoca.\\
Studi dinamici delle loro orbite indicano che i centaurs sono probabilmente degli oggetti con caratteristiche orbitali intermedie tra quelli della Kuiper Belt e le JFCs.\\
Sono probabilmente più simili a comete che ad asteroidi, e uno di questi, Chiron, è stato osservato avere attività cometaria.

\paragraph*{Trans Neptunian Objects}
Sono oggetti con $a\geq 30$ AU e sono divisi nei seguenti gruppi:

\qquad \textit{Kuiper Belt Objects}: si trovano tra 41 e 47 AU. A sua volta si divide nei sotto-gruppi dei \textit{Plutinos}, in risonanza 3:2 con Nettuno come Plutone e dei \textit{Cubewanos}, conosciuti come i classici KBOs.

\qquad \textit{Scattered-Disk Objects (SDOs)}: hanno orbite molto grandi e molto ellittiche, probabilmente dovute a un’interazione con Nettuno.

\qquad \textit{Oort Cloud Objects (OCOs)}: la Oort Cloud è una shell sferica di oggetti ghiacciati, di cui non si ha evidenza diretta perché troppo lontana e scura ma ipotizzata come luogo di origine delle comete a lungo/lunghissimo periodo. Si trova tra circa 50000 e 100000 AU.

\section{Potentially Hazardous Asteroids (PHA)}
Tra i NEO esiste una categoria di oggetti che riveste un ruolo importante: sono gli asteroidi potenzialmente pericolosi, Potentially Hazardous Asteroids (PHA). Rientrano in questa categoria tutti gli oggetti la cui minima distanza all'intersezione dell'orbita terrestre (Minimum Orbit Intersection Distance - MOID) è inferiore a 0.05 AU e la cui magnitudine assoluta H è minore di 22. Sono, quindi, quegli asteroidi che se impattassero con la Terra provocherebbero danni su larga scala, anche globale. Ad oggi si conoscono più di 2500 PHA, la maggior parte dei quali sono della famiglia Apollo, e ogni giorni vengo aggiunti o rimossi oggetti sulla base di nuovi risultati sulla base dei parametri orbitali.\\
La determinazione dell'orbita avviene sulla base delle osservazioni disponibili ed è definita da sei parametri orbitali, influenzati da diversi fattori. Quindi non si può determinare immediatamente con sufficiente accuratezza l'orbita ma servono osservazioni costanti nel tempo.\\
Al momento della scoperta di un PHA si determina l'orbita per i successivi 100 anni. Si possono determinare in tal modo i passaggi ravvicinati (fly-by) che l'asteroide avrà con la Terra e, conoscendo la regione di incertezza, la probabilità di collisione con la stessa.
I PHA con una probabilità non nulla di impatto con la Terra nei futuri 100 anni vengono catalogati nella Sentry Risk Table\footnote{https://cneos.jpl.nasa.gov/sentry/}, un sistema informatico gestito dal CNEOS (Center for Near-Earth Object Studies).\\
Oltre alla probabilità di collisione e alcune caratteristiche dell’oggetto (magnitudine assoluta, diametro) nella tabella vengono riportati i valori dell’asteroide sulla Scala Torino e Scala Palermo. Queste sono due classificazioni che permettono di quantificare il pericolo associato ad ogni asteroide: la Scala Torino è stata pensata per la comunicazione al pubblico, la Scala Palermo è più tecnica e viene usata direttamente dagli astronomi.
%%direi che ha poco senso mettere due paragrafi sulla scala milano e torino, ma non costa nulla

\chapter{Spettroscopia}

\section{Introduzione}
Lo studio della mineralogia superficiale di singoli asteroidi o di gruppi di asteroidi può fornire i dati per migliorare la nostra comprensione della loro origine ed evoluzione. La superficie degli asteroidi può essere studiata dall'interpretazione delle proprietà osservabili per determinare la presenza, l'abbondanza e la composizione mineralogica.\\
La spettroscopia di riflettanza nel visibile (Vis) e nell'infrarosso (IR) viene ampiamente usata per determinare la composizione degli asteroidi poiché può caratterizzare la composizione superficiale della maggior parte dei tipi di asteroidi. Le caratteristiche diagnostiche negli spettri, che derivano da transizioni elettroniche e vibrazionali all'interno dei minerali e delle molecole, sono riscontrabili nell'intervallo di frequenza $0.35-2.50\,\mu m$.\\
I minerali più importanti presenti negli spettri degli asteroidi sono: olivine, pirosseni, metalli Ferro-Nichel, feldspati e fillosilicati idrati e composti organici.\\
La maggior parte degli asteroidi è composta da un mix di questi minerali. Poiché i parametri spettrali dei diversi assorbimenti (i.e. la posizione delle bande e il rapporto tra esse) sono legati a una specifica composizione del singolo minerale, l'analisi spettrale della superficie degli asteroidi è in grado, nella maggior parte dei casi, di rilevare le firme mineralogiche caratteristiche di una particolare specie.\\
La possibilità di rilevare una feature dipende dall'abbondanza della particolare specie, in modo che la forza della feature possa essere rilevata sopra il rumore dello spettro.\\
In spettri di alta qualità può essere anche determinata la composizione e le relative abbondanze di un mix di minerali.\\
Per questi motivi l'analisi degli spettri può fornire una varietà di dati sulla composizione della superficie degli asteroidi.

\section{Spettro degli asteroidi}
Il flusso incidente che arriva sulla superficie di un asteroide è diviso in due contributi: la parte riflessa e la parte assorbita, il cui rapporto dipende dall'albedo. La parte assorbita riscalda la superficie del corpo. Questo emette, di conseguenza, radiazione di corpo nero determinata dalla temperatura raggiunta. Quindi la Terra riceve sia il flusso "solare" riflesso dall'oggetto che la radiazione di corpo nero.\\
Negli spettri degli asteroidi la radiazione solare riflessa domina nel range che va dall'ultravioletto (UV) al vicino infrarosso (NIR): $0.35-2.50\,\mu m$; mentre a lunghezze maggiori ($2.5-5.0\,\mu m$) il contributo della radiazione di corpo nero dell'asteroide diventa rilevabile.

\section{Tipi tassonomici}
La tassonomia è la classificazione degli oggetti in categorie definite da alcuni parametri caratterizzanti. Sin dagli anni '70 diversi autori hanno creato diverse classi tassonomiche basandosi su proprietà osservabili.\\
La classificazione tassonomica più utilizzata è quella di Tholen (1984), basata su più di 400 spettri ricavati dalla Eight-Color Asteroid Survey (ECAS), nel range spettrale $0.3-1.1\,\mu m$. Dagli spettri ECAS si possono distinguere undici classi spettrali (A, B, C, D, F, G, Q, R, S, T e V).\\
Barucci et al., nel 1987, usarono sette colori spettrofotometrici e gli albedo ricavati dalle survey IRAS per definire nove classi spettrali e per ognuna delle sottoclassi identificate dallo studio dettagliato degli spettri.\\
Infine, Bus e Binzel (Bus \& Binzel (2002b), Bus \& Binzel (2002a)) hanno derivato la loro classificazione tassonomica da dati spettrofotometrici recenti.

\subsection{Tassonomia di Tholen}

\subsection{Tassonomia di Bus}
Tassonomia di Bus
\section{Space Weathering (foorse)}

























\end{document}






